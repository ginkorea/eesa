\begin{tabular}{r{1cm} p{0.4in} r{1cm} p{0.4in}}
\toprule
sentiment & sentence & sentiment_score & explanation \\
\midrule
1 & It was just not a fun experience. & 0 & The overall sentiment of the statement is negative, as indicated by the use of words like "just not," "not fun," "not a fun experience," and the word "just" intensifying the negativity. \\
1 & Not a weekly haunt, but definitely a place to come back to every once in a while. & 0 & The overall sentiment of the explanations is slightly positive, indicating that the user finds the place enjoyable and worth revisiting occasionally. \\
1 & I don't have very many words to say about this place, but it does everything pretty well. & 0 & The overall sentiment of the text is slightly positive, as the user acknowledges that the place does everything pretty well, despite not having many words to describe it. \\
1 & I could care less... The interior is just beautiful. & 0 & The overall sentiment of the text is slightly positive, with an expression of admiration for the beautiful interior outweighing the potential negative connotation of the phrase "I could care less". \\
1 & Nice blanket of moz over top but i feel like this was done to cover up the subpar food. & 0 & The overall sentiment is negative as the mention of a "nice blanket of moz" implies good appearance, but the phrase "done to cover up the subpar food" reflects disappointment in the quality. \\
1 & So they performed. & 0 & The consensus among the explanations is that the statement "So they performed" lacks enough context to determine a clear sentiment, making it a neutral statement. \\
1 & Check it out. & 0 & The phrase "Check it out" generally has a slightly positive sentiment as it implies a suggestion or invitation to explore something, however, the lack of context makes it somewhat ambiguous. \\
1 & Bacon is hella salty. & 0 & The majority of the explanations imply a negative sentiment towards bacon, specifically highlighting its excessive saltiness, and express this sentiment with confidence. \\
1 & If you're not familiar, check it out. & 0 & The statement is neutral overall, with a sentiment score of 0.4, suggesting a slightly positive sentiment. The explanation is clear and provides a confident rating. \\
1 & Eclectic selection. & 0 & The phrase "eclectic selection" generally conveys a positive sentiment, as it suggests a diverse and carefully curated range of high-quality choices or options. \\
\bottomrule
\end{tabular}
