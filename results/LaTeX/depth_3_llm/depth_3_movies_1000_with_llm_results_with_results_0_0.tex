\begin{tabular}{r{1cm} p{0.4in} r{1cm} p{0.4in}}
\toprule
sentiment & sentence & sentiment_score & explanation \\
\midrule
0.000000 & susan granger's review of " session 9 " ( usa films )  sometimes you just get more than your bargained for . . . like when boston-based hazmat elimination , run by scottish actor peter mullan and his trusty assistant , david caruso , assures a town engineer ( paul guilifoyle ) that they can remove insidious asbestos fibers from a victorian hospital facility in a week .  erected in 1871 , deserted and decomposing since 1985 , the danvers mental hospital , is one of the most malevolent " locations " ever chosen for a film .  the structure is so massive - with its labyrinth of rubble-strewn corridors , collapsing floors , stagnant pools of water , isolation cells , and ominous surgical chambers where experimental pre-frontal lobotomies were performed - that their task seems impossible within that time frame .  and each member of their inexperienced crew ( stephan gevedon , brandon sexton iii , and josh lucas ) is coping with his own personal demons as , one by one , their minds seem to be affected by the grim areas in which they're working .  the film's title is derived from salvaged reel-to-reel audio-recorded sessions involving the demonic possession of a young woman who is suffering from multiple personalities .  by the time session 9 occurs so do dreadful disasters .  filmmaker brad anderson obviously envisioned this as a gruesome chainsaw-massacre-type ghost story but the script lacks structure and isn't particularly scary .  the conclusion is more ludicrous than convincing .  on the granger movie gauge of 1 to 10 , " session 9 " is a dark , gloomy 4 . silly me . . . at  first , i thought that the original name of the danvers lunatic asylum bore some reference to mrs . danvers , the creepy housekeeper played by judith anderson in alfred hitchcock's truly terrifying " rebecca " that also involved a cavernous mansion called manderley .   & 0 & The general consensus is that the film "Session 9" is slightly negative, with a lack of structure, scariness, and a conclusion deemed more ludicrous than convincing. The mention of multiple personalities and demonic possession adds to the overall negative sentiment. \\
0.000000 & the king and i , a warner brothers animated , musical feature , recycles the classic story of a woman who challenges the heart of a king , with obvious results .  when anna ( miranda richardson ) , a british schoolteacher , travels to saim to educate the king's ( martin vidnovic ) children , she learns that the king is treating his people unfairly , and must say something to the greedy ruler .  meanwhile , the king's prime minister ( ian richardson ) , the stereotypical villain , plots to overthrow the king , taking the throne .  the last , and most predictable , main subplot deals with the king's son ( allen d . hong ) , and his love for a servant , tuptim ( armi arabe ) , and how he conflicts with his feelings , and the ancient laws of saim .  not even the lone strong character of anna can save the unbelievably horrible waste of talent , as the king and i's problems could fill the blank pages of a journal .  i will only note the major difficulties , for it would take pages to elaborate on every detail .  the screenplay , written by arthur rankin , peter bakalian , jacqeline feather , and david seidler , which is based upon the play written by richard rodgers and oscar hammerstein , has some of the worst dialogue written in a film within recent memory , as every time the obnoxious king would shout , " etc , etc , etc , " i would cringe .  literally .  speaking of cringing- i did quite a bit of this during the rather short film , which is a classic display of terrible filmmaking .  besides the repetitive dialogue from the king , on the whole , the songs seem out of place , and unlike the lyrics , are unmagical .  the sole song which is used cleverly is " getting to know you , " which is used as anna shows the children the great outdoors , which they have never been exposed to .  unlike disney animated features , the king and i's songs don't add to the film , and are as uneffective as could be .  take the following scenario as an example , as the sheer horror of the king and i's music is at its worst .  you're being hunted by a dragon !  what do you do ! ?  sing a happy song !  martin vidnovic voices the king without effort or emotion- you hear the saying two negatives don't make a positive ?  believe it !  with the terrible dialogue that the king has , along with his awful voice track , the king is completely unbelievable , only shows mild signs of any personality , and the only thing that changes in the king is that he says " etc . , etc . , etc . , " more and more as the film progresses .  no personality at the beginning of the movie , none at the end .  and where does this character's personality change ?  hey , i thought anna was supposed to change him !  isn't that the whole plot ?  the prime minister's hideous sidekick ( darrell hammond ) brings his share of cringes as well - oh no !  another one of his teeth fell out !  hardy-har-har !  he is supposed to bring laughs for the kidlets , but even at age five i would have cringed while watching him .  by the way this review is going , you may think the reasoning for my hate for this film is due to not liking animated films- hence why i hate this movie , because the king and i is a disgrace to animation .  animated films , such as 1994's the lion king and 1998's the prince of egypt , are among my favorite movies of all time .  the animation team does design their share of well animated settings , so this makes it easier to take my mind off of the annoying king , until i realize that day and night switch back in fourth within seconds .  i have not read the play , or seen the oscar winning , 1956 film adaption , but from what i can tell , the screenplay for the 1999 version completely butchers the play , for the king and i is never magical , nor interesting .  if it wasn't for miranda richardson , who voices anna with feeling , the king and i could earn the title , " worst movie of the decade . "  instead , the king and i will just go down as among the year's worst .  the bottom line- avoid this movie at all costs .  not even young children , the target audience in this film , will enjoy it .  not the slightest bit .   & 0 & The reviewer expressed a strong dislike towards "The King and I," criticizing various aspects such as screenplay, dialogue, songs, characters, and overall storytelling, resulting in a negative sentiment towards the film. \\
0.000000 & if you're the kind of person who goes to see movies just because you long for some of that overpriced theatre popcorn ( butter optional ) , then this is the movie for you !  indeed , this has got to be either one of the most unimaginative rip-offs of other recent action movies , or an incredibly unfunny spoof of them .  it's difficult to fathom such insipidness unless you actually watch this film .  but at least we're warned very quickly that we may regret our ticket purchase , giving us an opportunity to sneak on out and into the adjacent show .  what are the four ingredients of a really bad action movie ?  first , the movie gives us an introductory premise .  huge caverns exist deep beneath the ocean floor , and in this area , many ships have disappeared .  oooh . . . scary !  secondly , a cheesy soundtrack tries to connote a tone of mystery , but only succeeds in drowning our ears with an abrasive musical score .  third , the main character is a mercenary that delivers goods without asking about the contents of his cargo ( treat williams ) .  he operates a sophisticated military-style pt boat and every word that comes out of his mouth is awash in comical flippancy .  finally , in the cargo hold , we see those that hired him .  they are also mercenaries that have tough-looking haircuts , talk with accents , and try to show how macho they are .  during their trip across the stormy sea , their boat suffers an incident and requires repairs .  spotting a cruise ship in the distance , they make their way to the ocean liner and devise a plan to raid the machine shop , take the parts that they need , and then continue on their merry way .  little do they know that this cruise ship has become infested by some kind of ocean monster .  yet , they board the ship armed to the hilt with grenades and machine guns that can kill dozens in a matter of seconds .  this is nothing more than a by-the-book action film .  their realization of the situation that they're in doesn't happen until they are in the bowels of the boat .  those who are dumb enough to stray off on their own will ultimately get killed .  the corridors on the ship are narrow , misty , and provide the kind of atmosphere that all scare-fests must have .  yet , despite the predictable nature of this film , there are some scary " boo " moments .  but most of it is just outright silly .  and this film is unusually gory too .  monsters basically suck off the flesh and spit out skeletal remains .  there is one particularly neat scene where a monster has been cut apart and reveals a victim that is still alive .  he screams horribly as the monster's digestive juices continue to slowly eat him away .  additionally , it borrows heavily from speed2 , alien and a bunch of other recent films .  the mercenaries even run into a lone , surviving passenger ( femke jannsen ) who looks amazingly like sandra bullock .  to be honest , when i left the theatre , i just had to laugh at how witless the film was .  it's moronic fun at best .  so , if you're hankering for a large bucket of popcorn served with a side of silliness , then this might just hit the spot .   & 0 & The overall sentiment towards the movie is negative, as the author criticizes it for being unimaginative, unfunny, and insipid, while also mentioning its predictability and heavy borrowing from other films. However, they do mention that it may provide some mindless fun for those seeking a silly experience. \\
0.000000 & i didn't come into city of angels expecting greatness .  i've never seen wim wenders' wings of desire , the classic movie upon which city is loosely based .  then again , i have seen enough stories which are based upon a similar plot device , with the little mermaid ( both the disney version and the original folktale ) being among them , that i had some high expectations about the possible power such a story of impossible love can hold .  unfortunately , city of angels ended up fulfilling few of them .  the plot , for those that couldn't tell from the previews , revolves around the angel seth , played with an almost creepy intensity by nicolas cage , who , in the midst of his angelic duties , falls in love with a heart surgeon named maggie ( meg ryan in her most endearing performance since when harry met sally ) .  of course , his being an angel prevents him from doing much about his love except appearing at random times to talk to her , watch her buy groceries , only to disappear in the blink of an eye .  their love must remain unrequited unless seth decides to make the ultimate sacrifice and become human .  using this framework as a jumping-off point , the movie attempts to veer through some heavy philosophical ruminations on the nature of desire , the joys of being human , and the definition of perfection .  the first half of the movie succeeds on most points .  cage excellently plays the eminently difficult role of an angel who doesn't know feelings so can't really express , preventing the character of seth from getting boring despite his limited repertoire of intent looks and hang-dog expressions .  unfortunately , cage takes the intensity too far sometimes , and then seth comes across as more creepy than sensitive .  as maggie , ryan manages to be convincing as a heart surgeon who has trouble coming to terms with her having lost a patient on the operating table despite having done everything right .  her beauty , unlike her unbearable cuteness in french kiss , is mature , intelligent , and winning .  likewise , some interesting ideas float around at the beginning of the film .  when the camera pans through traffic jams and libraries and we get to hear the thoughts of the random people who flash across the screen , the audience experiences a little of what it must be like to be an angel .  the beautiful camera work , shooting down onto the hectic world of los angeles from the improbable perches of the angels , also gives us a sense of the unique wonder angels feel .  the film begins to lose its way , though , when the focus tightens more and more on seth and maggie .  the grand , angelic perspective gets lost , except for some idly tossed lines about the incredible beauty of the world through an angel's eyes .  the movie devolves into an examination of how seth can't feel the world or , more importantly , he can't feel maggie : he can't smell her hair , feel her touch , or taste the pears she eats .  this change in focus attempts to capture the audience in seth's intense longing , but in doing so , the conflict disappears .  if he wants maggie so badly , then why doesn't her just make the leap and become human ?  after all , what's so great about being an angel ?  sure , you get to sit on marlboro signs , but what's that compared to getting to be with meg ryan ?  and from there , once the yearning has been established and the romantic denouement must occur , it's all downhill .  the philosophy becomes heavy-handed , the dialogue pedestrian when it tries to be deep , and the plot twists simply attempt to yank a few more tears into the audience's hankie .  it's the last thirty minutes of the movie , then , that wrecks the film .  i feel like the writer , by pulling out all the melodramatic stops , has robbed me of what could have been a genuinely powerful movie experience on both the romantic and the philosophical level .  i came out feeling robbed , seeing so much possibility in a film becoming nothing .  i could go on longer , but i don't want to " ruin " the end by revealing any of the cheap plot devices the film relies on .  city of angels , then , is a paradox .  it's a well-acted ( particularly look for dennis franz cutting against type-casting as a happy-go-lucky fallen angel ) , well-filmed , and based on a wonderful idea .  all these possibilities , though , are what make the film so unredeemable when the closing credits come up .   & 0 & The user has a negative sentiment towards the movie "City of Angels", feeling disappointed and robbed of a potentially powerful movie experience due to high expectations not being met, the plot being disappointing, dialogue being heavy-handed, and a loss of the grand angelic perspective. \\
0.000000 & what are we going to do with jim carrey ?  viewers of television's " in living color " know this one-man cartoon from such characters as fire marshall bill .  viewers also know that " in living color " is a skit-show and that a little of jim carrey goes a long way .  unfortunately , this fact was forgotten by the makers of the carrey comedy ace ventura : pet detective .  three writers , including carrey , worked on the slapstick story , which sends a self-styled " pet detective " on the trail of a stolen dolphin .  the missing mammal belongs to the miami dolphins , who need their mascot for the upcoming superbowl .  for plot porpoises , this story works as well as any three stooges short .  carrey gets to do his " official " schtick as he snoops around greater miami .  he leers and sneers , craning his neck to funny effect .  he even does his captain kirk impersonation .  again .  all of this is pretty harmless stuff up until the point that you realize that the writers have absolutely no intention of focusing on anyone * other * than carrey .   ( suggested alternate title--jim carrey : will do anything for a laugh . )  export it to france and you may have a hit .  as it stands , ace ventura isn't even good kid's stuff .  the profanity count , alone , is too high .  which is ironic , since children are , probably , carrey's best audience .  the film doesn't even have the goofball charm of chris elliott's recent no-brainer cabin boy .  sure , carrey has his moments .  but what can you say about a film whose high-points include watching carrey slink around to the theme from " mission impossible ? "  ace ventura has one glaring incongruity .  amid the butt-jokes and double-takes , the script takes great pains to set-up an elaborate and rather funny " crying game " gag .  and , for * this * intended audience , that takes ( ahem ) cojones .   & 0 & The text generally expresses a negative sentiment towards the film "Ace Ventura: Pet Detective" due to criticism of its excessive focus on Jim Carrey, high profanity count, lack of charm compared to other films, and an incongruous "Crying Game" gag. \\
0.000000 & she was the femme in " la femme nikita . "  he was the baldwin in " backdraft , " " sliver , " and " fair game " ( with cindy crawford ) .  together , anne parillaud and william baldwin conspire to make " shattered image " the biggest piece of hooey since the stallone/stone " thriller " " the specialist . "  the film poses the question " what if the life you're living now is really a dream , and your dreams reality ? "  it's either about a woman who's haunted by a recurring ( and recurring and recurring ) nightmare that she's a hired assassin , or it's about a hired assassin who's haunted by a recurring ( and recurring and recurring ) nightmare that she's honeymooning with william baldwin in jamaica .  it doesn't much matter and believe me by the time " shattered image " runs its painful and pedestrian course you won't care .  these two lives , with parillaud looking like siouxsie sioux with a black wig , black emotionless eyes , and black leather clothing in the seattle-based assassin scenes , and moping around like karen carpenter in the jamaica scenes , play out endlessly throughout the film and the result is it's now * twice * as boring as it might have been .  it's not that complicated plots can't be entertaining .  of course it helps if you have interesting characters , crisp dialogue , and a modicum of style .   " shattered image " isn't complex , it's just stupid .  and boring .  parillaud and baldwin , who aren't exactly shakespearean material to begin with , are saddled with such leaden dialogue that their characters have zero chance of breaking free of their cardboard confines .  lines like :   " you don't beg , you insist .  i like that in a woman . "  that's parillaud's character talking . . .  to her cat !  and :   " you're not the reason i couldn't care less about you . "  huh ?  and this wonderful bathroom interchange early in the film :   " if you give me a couple of minutes you know i'll charm the pants off you . "   " i don't have that kind of time . "  talking of pants , parillaud has her clothes off faster than you can say " point of no return . "  we have come to expect this from billy baldwin , but it might have been nice to have learned something about their characters first .  but there's nothing to learn .  karen is as interesting as a cereal box , a someone's-out-to-get-me crybaby who imagines the voice at the other end of the phone , the stranger who sends her flowers , maybe even her husband himself , is her would-be killer .  siouxsie is the chromium cool , tough-as-nails crack killing machine who shoots out a couple of mirrors in order to justify the film's meaningless stock title .  baldwin seems more interested in parillaud's nest egg ( so that he can pave paradise and put up a parking lot ) than he does in her .  each time graham greene shows up he gets killed .  barbet schroeder ( " reversal of fortune " ) co-produced and should be ashamed of himself .  every now and again it's fun to watch a really bad movie .  and every now and again , as " shattered image " makes agonizingly clear , it isn't .   & 0 & The film "Shattered Image" is heavily criticized for its boring plot, uninteresting characters, and poor dialogue, with the explanation suggesting it fails to deliver an engaging story or memorable performances, resulting in an overall negative sentiment. \\
0.000000 & stars : armand assante ( mike hammer ) , barbara carrera ( dr .  charlotte bennett ) , laurene landon ( velda ) , alan king ( charles kalecki ) , geoffrey lewis ( joe butler ) , paul sorvino ( detective pat chambers ) , judson scott ( charles hendricks ) , barry snider ( romero ) , julia barr ( norma childs ) / mpaa rating : r / review :  in the 1982 updating of mickey spillane's 1947 novel " i , the jury , " hard-boiled detective mike hammer is a vietnam vet who drives a shiny bronze trans am , dresses like don johnson in " miami vice " with less pastels , and has sworn off alcohol .  however , he still smokes his lucky strikes , detests all forms of authority , and kills at a whim .  beyond that , the updated film retains little or no resemblance to the original pulpy page-turner by spillane , probably the most infamous and often reviled of all mystery writers .  the movie starts off with a bang : a howler of an opening credits sequence that is a cheap steal from the james bond series , complete with cheesy graphics and an overbearing jazz score by bill conti ( " rocky " ) .  after that , the movie and the book begin the same , with the murder of jack williams ( frederick downs ) , a one-armed detective and hammer's best friend .  hammer declares that he will seek vengeance for jack's death , and with the help of his devoted secretary , the blond and shapely velda ( laurene landon ) , and the alternately friendly / antagonistic police chief pat chambers ( paul sorvino ) , he is immediately on the killer's trail .  here the movie splits completely from the book , and dives into a convoluted and improbable tale of government conspiracy and mind control tactics involving the mafia , the cia , one of hammer's vietnam vet buddies , and a kinky sex clinic .  many of the same characters from the book appear in the movie , but they take on slightly different roles .  for instance , charles kalecki ( alan king ) , a numbers runner and narcotics dealer in the book , turns into a suave mob boss .  and , more importantly , hammer's suspicious love interest , charlotte bennett ( barbara carrera ) , morphs from a run-of-the-mill psychiatrist into the coordinator and founder of the sex clinic .   " i , the jury " is one of several cinematic renditions of spillane's books ( including a 1953 version which was made in 3-d ) , but this film differs from those earlier versions in one major way : it includes all of the sex and violence spillane wrote about that could never be given screen treatment due to hollywood's production code .  although this takes the 1982 version of " i , the jury " closer to the core of the original subject matter , it is in this aspect that the film received the most criticism , because it took this new license to extremes that many argued surpassed what was in the book .  rest assured , the movie not only includes a great deal of nudity , but it is thoroughly violent , especially toward women .  it features one woman having her neck slashed , a set of twins forced to strip before being stabbed to death by a psychotic sexual deviant programmed by the cia ( judson scott ) , and another woman shot point-blank in the belly by hammer himself .  no one would deny that spillane's writing has a definite misogynistic nature , but the movie seems to take it a step further by giving it such glorious screen treatment ; its constant equation of sex and violence , much of which is played with the intention of being erotic , is quite unsettling .  it's no surprise that the movie , like the book , fades to black with a dead woman on the floor .   " i , the jury " had a troubled production and was not well-supported by the studio that made it , which is one explanation why it didn't do well in theaters and many people have forgotten that it was ever made .  the script was written by larry cohen , who is best known for his creatively cheesy but nonetheless effective monster movies , like " it's alive " ( 1974 ) and its two sequels , " q " ( 1981 ) , and " the stuff " ( 1985 ) .  cohen wrote the script thinking he was going to helm the project as well , but he was yanked from the director's chair after only a week's worth of shooting because he was already $100 , 000 over budget .  he was quickly replaced by richard t . heffron , who has worked for the last three decades on a handful of undistinguished movies and dozens of television projects .  heffron was obviously brought in not for his talent , but because he could make the movie rapidly and efficiently .  it shows in the final product .  cohen had personal interest in the updated version of hammer , but heffron has none .  he shot the movie quickly and clumsily , and although some scenes ring true , most of them are flat , trite , and invariably dull .  the movie features numerous car chases , shoot-outs , and stunts , but heffron's background in television is the dominant tone ; despite the graphic violence and full-frontal nudity , " i , the jury , " takes on the air of a made-for-tv quickie , with no real punch or depth .  but the problems in " i , the jury " run deeper than the technical .  the central fault in this updating is mike hammer , whose character was lost in the shuffle while updating from the fifties to the eighties .  because spillane wrote all his hammer mysteries in the first person , hammer's character is central to the tale because all the events are filtered through his persona .  we never really get that impression in the movie -- there is no first-person voice-over narration and some scenes don't have hammer in them at all .  consequently , a great deal of the texture of spillane's storytelling is lost .  the period updating turns out to be a detrimentally bad idea because much of hammer's moral code is thrown to the wind .  despite his characterization as a hard-nosed , violent , misogynistic killer , hammer always stuck fervently to his own moral code .  the title itself , " i , the jury , " refers to his anti-establishment notion of being his own law .  unlike private eyes who seek out the bad guys and then turn them over the police , hammer both pursues the criminal and exacts the punishment .  in this way , he can be seen as " above the law , " but he still adheres strictly to her own personal code of conduct , his own morality .  the movie forgoes that aspect of his character , and hammer comes off not only as amoral in society's terms , but in any terms , especially his own .  if anything , hammer always had his professionalism , but the movie does away with that in the first three minutes by showing him rolling in the sack with the wife of a client who had paid him to find out if that wife was being unfaithful .  maybe the scene was intended for laughs , but it only cheapens hammer's character and is , by all accounts , a lousy way to start the movie .  the blame for hammer's character can't be laid on assante's shoulders , because despite some unnecessary marlon brandon-like mumbling , he delivers a fine performance .  spillane never once described hammer's physical attributes in any of the dozen books in which he appeared , so any actor could conceivably portray him .  of course , because of the lack of written description , those who have read spillane's books will have a strong personal notion of what hammer looks like , and therefore almost any screen incarnation will somehow fall short of expectations ( spillane , who played the character himself in 1963's " the girl hunters , " is generally considered the best of the film hammers ) .  the rest of actors are most un-noteworthy .  with the exceptions of alan king and paul sorvino , everyone who appeared in " i , the jury " were up-and-comers who basically went nowhere .  many of them ended up working in television ( like carrera , who had a short stint on " dallas " in the mid-eighties ) , which only adds to the made-for-tv atmosphere of the film .  maybe someday , someone will manage to get the right elements together and make an effective film rendition of a spillane book , but this is certainly not it .   & 0 & Overall, the sentiment of the text is slightly negative, as the movie is criticized for deviating from the original source material, excessive violence and nudity, poor direction, a flawed portrayal of the main character, disappointment with the cast, and an unsatisfactory overall atmosphere. \\
0.000000 & by trying to satisfy every kind of viewer , it's possible that sphere may end up pleasing no one .  action lovers will be bored by what they will see as an interminably boring setup .  audience members who crave more intellectual fare will be disgusted by the film's sudden collapse into mindless storytelling and by the ending , which is an insulting cop-out .  somewhere out there , maybe there's a small cadre of film-goers who will appreciate sphere's dubious charms , but i'm not among them .  i sincerely hope the novel is better than the movie ( i no longer read anything by either michael crichton or john grisham ) , because if the finished motion picture product is anything to go by , it's hard to understand why the rights were optioned .  sphere is the kind of first- class mess that only a top-line director with an a-list cast can create .  with expectations high ( and how could they not be , considering that another barry levinson/dustin hoffman collaboration , the excellent wag the dog , is still playing in theaters ? ) ,  something this bad can't help but look even worse .  the last time a big-name , big-budget film displayed this level of ineptitude was last year's batman & robin , and everyone knows how that movie was received .  sphere starts out a little like an amalgamation of contact and james cameron's the abyss , but , somewhere along the way , it collapses into the cellar with another recent science fiction effort , event horizon .  science and philosophy , which are used to good effect during sphere's first hour , give way to mindless , confusing action sequences .  attempts at characterization fall apart .  intelligent writing , which is evident early on , is replaced by hackneyed drivel .  special effects take over as the plotline devolves into incoherent silliness .  but all that is just in preparation for the ending , which is inexcusably awful .  this is the time-honored deus ex machina device used to its worst effect .  i left the theater feeling cheated by the way crichton and his screenwriters had chosen to end the film .  there is some promise , but it's all in the setup .  we're introduced to norman goodman ( dustin hoffman ) , a psychologist who once wrote a $35 , 000 report for the government about what to do in the event that a crashed space ship is discovered .  when one is found in the middle of nowhere , 1000 feet below the surface of the pacific ocean , norman is called in to be part of the welcoming committee .  on the team with him are beth halperin ( sharon stone ) , a biochemist who was once his student and lover ; harry adams ( samuel l . jackson ) , a mathematician who earned his first doctorate at the age of 18 ; ted fielding ( liev schreiber ) , an astrophysicist who is awed by the opportunity to explore alien technology ; and harold barnes ( peter coyote ) , the government operative in charge of the mission .  together , the five descend into the bowels of the ocean , where they rendezvous with a temporary sea base on the ocean floor from which they will attempt to make first contact .  for a while , sphere had me fooled into thinking it was going to take an astute approach to the man-meets-alien situation .  the overall scenario is not without promise and several plot twists ( such as the revelation that the enormous craft is actually an american space ship , apparently from the future ) offer intriguing possibilities .  then , right around the one-hour mark ( that's the time to sneak into the theater next door and check out whatever's left of titanic ) , the virtually non-stop action begins , and , once it starts , the script becomes superfluous .  this might be acceptable if director levinson generated some legitimate tension , but , instead , he relies on loud , overbearing music , strange camera angles , and quick cuts to make things " exciting . "  additionally , because none of the characters are well-formed ( a common failing in anything penned by crichton , who's more interested in technology than people ) , viewers don't develop much of a rooting interest .  it makes sitting through sphere a frustrating and pointless experience .  what about that a-list cast ?  not surprisingly , the most energetic performance is given by samuel l . jackson , but his harry isn't a person ; he's a walking plot device spouting occasionally-witty dialogue .  dustin hoffman isn't lively or particularly good -- it's ironic that this , which may be his worst work in a decade , has arrived in theaters on the heels of his best actor nomination ( for wag the dog ) .  sharon stone and peter coyote are both flat .  their characters exhibit little evidence of emotion ; automatons would have been as effective .  then there's queen latifa , who , despite getting fifth billing in the credits ( ahead of liev schreiber , who boasts at least quadruple her screen time ) , has less than a handful of lines and almost nothing to do other than inflate the body count .  i like to think that levinson and hoffman , recognizing how uninspired this movie was likely to be , chose to make wag the dog as a sort of penance ( the low-budget picture was filmed during sphere's lengthy pre-production phase ) .  if that's the case , forgiveness is granted .  i'm less inclined to look favorably upon crichton , although he has a few enjoyable titles on his resume ( jurassic park and the levinson-directed disclosure come to mind ) .  even if his novel was butchered in the adaptation process , crichton's credit as a producer disallows him absolution .  he was a willing participant in a creative travesty .  no wonder sphere is being released in february , in the midst of the early year's cinematic wasteland .  it deserves no better than to get sunk by the unstoppable titanic , which should plow sphere under on its way to a ninth-consecutive weekend atop the box office heap .   & 0 & The text provides extremely negative feedback on the movie "Sphere," criticizing its lack of appeal, mindless storytelling, and insulting ending, as well as expressing disappointment with the direction, writing, characterization, performances, and Michael Crichton's involvement. Overall, the sentiment towards the movie is overwhelmingly negative, with multiple reasons contributing to this negative sentiment. \\
0.000000 & the premise of this movie is , well , pretty far-fetched .  tom berenger plays shale , a mercenary who is temporarily out of work ( those fools at the cia have denied his existence just because he and his buddies botched a job in cuba ) .  fortunately , his girl friend ( diane venora ) , a teacher at christopher columbus high school in miami , gets her knee cap broken by a disgruntled student , creating a job opening for shale as a substitute teacher .  not telling his girl friend , who might object on pedagogical grounds , he creates a number of fake higher degrees for himself ( from yale , harvard , princeton , et al ) and begins his tenure as a high school teacher .  the students ( junkies , drug dealers , gang members , sleazy sluts , ice-pick wielders . . . you  get the picture ) don't really take to him right away , so he hits one in the face with a can and breaks a few fingers .  this gets their attention to a certain extent , so he tells them the story of the vietnam war : " see , some homeboys from the north tried to muscle in on the turf of the homeboys from the south . "  oh yeah , now they can dig it ; the problem is just that nobody ever explained it properly before .  but wait !  there are drugs being dealt in the school itself !  and behind the whole scheme , in cahoots with the head gang , the kod ( no , not " cod " , but " knights of destruction " . . . really ! ) , is none other than . . . the  upright , ex-cop principal , played by the forgotten ghostbuster , ernie hudson !  so shale does what any good teacher would do .  he gets his buddies together , they gather together a bunch of bazookas and other major weapons , explosives , and cool stuff like that , and they have a big showdown against the drug dealers and kod at the high school .  ok , so the premise is not just far-fetched , it's downright dumb .  if this were a hong kong action comedy , we might just accept it , but it takes itself far too seriously to be truly fun .  oh , it has its moments ; how one can truly hate a movie in which huge ( really huge ) amounts of cocaine are delivered in school busses ?  and to be fair , it is almost never really boring , as the action is interrupted by only short sequences of actual story .  but over all , this is pretty much a made-for-tv movie with more ( and bigger ) explosions and more foul language .  in fact , it reminded me of " miami vice " without the production values , babes in skimpy bikinis , and pastels .  if you can sneak into the theater without paying , go for it .  otherwise , wait for video .  the flying inkpot rating system : * wait for the tv2 broadcast .   * * a little creaky , but still better than staying at home with gotcha !   * * * pretty good , bring a friend .   * * * * amazing , potent stuff .   * * * * * perfection .  see it twice .   & 0 & The explanations highlight various negative aspects of the movie such as its far-fetched and dumb premise, lack of production values, interruption of action with short sequences of story. These criticisms contribute to the overall negative sentiment score. \\
0.000000 & robin williams has the rarest of gifts : the ability to rise above the most inept material and suffuse it with his irreverent style .  overwhelmingly , his worst films are pleasantly diverting at worst and enjoyable at best ( with the notable exception of flubber ) .  so when i , the one person who has refused to abandon him despite patch adams , tell you that not even williams can save his latest project , you know it's in trouble .  jakob the liar is a confused , muddled little movie ; a generically " uplifting " film with a fundamental contradiction : the message it delivers is depressing as opposed to inspiring and the movie doesn't realize it .  williams plays jakob haim , a jew imprisoned in a polish ghetto during world war ii .  one night he wanders outside after curfew and is promptly sent to the office of a high-ranking german officer for his punishment .  jakob gets off easy and he gets to hear approximately 30 seconds of a radio broadcast .  the announcement ( in english , but punctuated by a triumphant " heil hitler ! " ) is that russian troops are only miles away from jakob's ghetto .  liberation ! he thinks .  the next day , jakob tells the news to his closest friend , a volatile prize fighter named misha ( liev schreiber ) , who despite being sworn to secrecy passes the message along .  soon , it is a common assumption in the ghetto that jakob has a radio hidden in his home -- a crime punishable by death .  this is absurdly false , but the more jakob tells the people of the ghetto this , the more convinced they become that he is abreast of the latest developments in the war that is to decide their fate .  the danger , of course , is that the germans allegedly have informants throughout the ghetto , and rumors about the radio can get out and put jakob in great danger .  in a curiously irrelevant subplot , jakob finds a 11-year old girl who has separated from her parents and decides to hide her in his small home .  apparently he is afraid that she will be discovered , and goes to great length to make sure of that -- just why the idea frightens him is never made clear .  he and the girl build an uninvolving , generic relationship that never goes anywhere and is as irrelevant at the end as it was in the beginning .  the moral of the story is that hope is the best medicine .  but jakob the liar forgets that the hope that jakob brings to his ghetto is false .  such hope inevitably leads to expectations of its realization and when those expectations aren't met , the results are far worse than if there was no hope .  that's the film's biggest detriment : it is based on a false assumption and thus comes off as painfully false .  it's never moving because it doesn't give us anything to be moved by .  jakob the liar fails to tug any heartstrings which destroys its purpose for existing .  robin williams doesn't inject the film with life , as a matter of fact , he seems a little out of it , as if crippled by his fake accent .  he is stragely unenthusiastic ; his character is one who spreads hope but his performance is lifeless , hopeless .  not since good will hunting has he abandoned his signature style to this extent ; this isn't the robin williams we know and love .  in good will hunting he became a serious actor and won an oscar for it .  here , he is more of a wannabe serious actor ; an impostor .  jakob the liar will be compared with last year's life is beautiful because it is being marketed as a " holocaust comedy " .  it's not a comedy .  it tries , sometimes , but it rarely works .  is it a melodrama ?  a war movie ?  a character study ?  no , no , and no .  jakob the liar is the kind of movie that can't be placed into a category ; not because it covers so many different genres but because it fails at just about every one it attempts .   & 0 & The sentiment of the text is predominantly negative as the author expresses disappointment with Robin Williams' performance in "Jakob the Liar", criticizes the film for being confusing and painfully false, and unfavorably compares it to "Good Will Hunting" and "Life is Beautiful". \\
\bottomrule
\end{tabular}
