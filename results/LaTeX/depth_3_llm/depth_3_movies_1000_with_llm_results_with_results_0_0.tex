\begin{tabular}{rlrrrlr}
\toprule
sentiment & sentence & sentiment_score & confidence_rating & explanation_score & explanation & results \\
\midrule
0.000000 & battlefield long , boring and just plain stupid  battlefield earth  a film review by michael redman copyright 2000 by michael redman  in my mid-teen years , i had a horrendous re-occurring nightmare .  behind the wheel of a car , i was driving down a straight road in the middle of a desert .  no scenery except the horizon line and the converging parallel lines of the highway .  no matter what i did , the view didn't change : travelling but not getting anywhere .  each time i awoke in a sweat , terrified .  you don't have to be carl jung to understand that dream .  powerless to make changes , trapped in a boring situation with no hope of rescue , this is the stuff of nightmares whether we are asleep or awake .  this is exactly how you will feel 15 minutes after " battlefield earth " begins .  for all of its flash and style , l . ron hubbard's science fiction epic is the earliest and best entry for the dullest summer film of 2000 .  dull " and " stupid .  in the year 3000 , the psychlo aliens have ruled our planet for 1 , 000 years .  humans either work as slave labor in mining operations or live as barbarians .  there's no hope .  the future is bleak .  then the psychlo make a mistake and capture feisty jonnie goodboy tyler ( barry pepper ) who organizes a revolution against chief of security terl ( john travolta ) and the alien race .  it's an archetypal post-apocalyptic plot full of promise .  considering some of the talent involved and all the millions thrown at the screen , it's difficult to see how it could have failed so miserably .  but it does .  the story is so full of holes that it falls apart within minutes of the opening credits .  are we really supposed to believe that after a millennium of looking for gold , the psychlo never discovered fort knox ?  or that fighter planes are still in pristine condition after all that time -- and gassed up ?  or that the cavemen become such expert pilots in seven days that they can easily down the advanced alien ships ?  or that the psychlo spy cameras somehow don't notice that their slaves are missing for days ?  travolta prances across the screen hamming it up for all he's worth and is almost entertaining .  almost .  the rest of the actors are wooden mannequins trying not to laugh while delivering lines that no person -- " man-animal " or not -- would ever utter .  some of the film _looks_ good , but it also looks so familiar .  a race of large hulking ape-like creatures has taken over the world while our cities lie in ruin .  sound like something charlton heston might be in ?  the psychlo look like overweight klingons in gear from " dune " .  the final air battle between the air force fighters and the psychlo ships in their high-tech city is something george lucas might have been associated with .  it's too loud .  it's too oppressive .  it's too slow .  it's too long . . . far  too long .  and on it goes .  the list of the problems with " battlefield earth " is endless .  it's difficult to find anything in the film that does work .  oh yeah .  the color scheme is nice .  the real question is how this movie ever got made .  could it have something to do with the fact that l . ron hubbard was the founder of scientology ?  and that john travolta is a member of the church .  that might explain why travolta bought the rights to the novel years ago .  but it doesn't give us a clue as to why first-time screenwriter corey mandell's atrocious script was used .  or why the high-profile project was entrusted with roger christian who had never before directed a major film .  or why no one looked at this thing before it was released and realized there might be problems .  the one bright spot is that no longer will kevin costner's " the postman " ( which i begrudgingly admit as a guilty pleasure ) be " the " big budget science fiction failure .   " battlefield earth " has the honor sewed up .  often reviewers will recommend that you skip mediocre films and wait for the video .  that's not the case here .  you can act now and make the decision to not see it on the big screen or on the small one .  don't hesitate .  strike now while the iron is cold .   & 0 & 0.933333 & 0.900000 & The summaries of the explanations indicate that the text consistently expresses an overwhelmingly negative sentiment towards the film "Battlefield Earth," using highly negative language and descriptors to criticize the plot, acting, direction, and overall quality. & 0.000000 \\
0.000000 & there are some pretty impressive stars in lost in space - it's just that none of them happen to be actors .  the stars i'm referring to are the computer generated ones that make up the movie's " outer space " ; the stars that the less impressive actors hurtle thru as they try to find their way home .  yes , in terms of acting , the star power is .   .   .  well , more like a falling star .  kinda pretty , short-lived , and pretty much dead .  lost in space , as if you really didn't know , is based on the 60's sci-fi television series of the same name .  it's the year 2058 , and earth's precious resources are quickly being usurped by the needs of its massive population .  john robinson ( william hurt ) is the scientist leading a mission program to colonize a foreign planet .  earth's entire existance is contingent upon a successful mission , but nobody , whether it be john's family or hired battle pilot don west ( matt leblanc ) , seems too enthusiastic about leaving their home planet for several years .  john's family consists of his wife maureen ( mimi rogers ) , his atypical teenage daughter penny ( lacey chabert ) , his ingenious son will ( jack johnson ) , and his beautiful scientist daughter judy ( heather graham ) .  despite everyone's reluctance , the jupiter 2 spacecraft abandons planet earth and makes its way into the vast eternity of space .  unknown to anyone but the audience , there is an evil doctor stowaway determined to sabotage the entire mission .  dr . smith ( gary oldman ) has been hired by a group of rebel conspirators to turn the expedition sour , and dr . smith has re-programmed a talking robot to " destroy robinson family " !  when everything that could go wrong does go wrong for both sides , the spaceship is warped to an unknown destination , and now the premise of being lost in space is complete .  as for the audience , you will likely be lost in boredom by this point , wondering if the plot , like the jupiter 2 , will ever get off the ground .  it's hard to tell who deserves the most blame - the incredibly bland and corny characters or the horrifically lame script ?  chabert is basically the only one to overdo it , sounding like a whiny munchkin on helium .  if you , like me , were convinced by commercials that her voice was altered for some sort of plot twist where her body would be taken over by aliens , you're wrong !  that's just her normal voice !  in yet another example of a " friend " faltering on the big screen , leblanc is so incredibly dull and yet so obviously trying so hard to be so incredibly charming ( make sense ? ) , it makes you want to shove his ass out the escape pod corridor without an escape pod .  graham is a babe - thank goodness there was something for me to think about during this film .  hurt , the black hole of excitement , sucks up any energy that might have been left .  if hurt were available in tablet form , he would be a prescription strength sleeping pill .  johnson isn't dull , he's just lame as the young know it all who winds up saving everybody's ass all the time .  want an example of how cool this kid can be ?  how about when he convinces the robot to think with its heart and reconsider killing the family ?  hey , don't laugh - the thing actually listened to the mr . rogers-would-be-proud sentiment .  but alas , if you thought a character couldn't be much worse , there was rogers as the epitome of generic ( or , mother as she was known ) .  why hire an actress ?  they could've had a white cardboard cutout with the word mom printed on it .  now that would've had some pizzazz !  lost in space luckily doesn't suffer in every single category that it could have .  the special effects are crisp , clear , and at least mildly captivating , unlike any of the presences onscreen save it be oldman , who plays his evil character with a great deal of fun and finesse .  unfortunately , oldman is locked away for most of the film , giving us nothing but ample mocking opportunities to enjoy .  while the special effects are pleasing to the eye , they are nothing you couldn't find in most modern sci-fi films .  contact , for example , far exceeds this film in terms of imagery and imagination .  lost in space just has too many shortcomings to ever be considered a work of cinematic art , with numerous contradictions ( the time travel aspect was horribly flawed ! ) , wooden and corny acting , worse dialogue , and an ending so disappointing , you'd be happier to have seen the entire robinson family get blown to smithereens .  then again , with an ending like this film has , it's obvious a sequel is already being considered .  what an awful note to end on , knowing there could be more of this in a year or two .  the attempt to be family oriented is commendable , but lost in space is lost with the illusion that special effects and the nostalgia of a classic tv series being revisited is enough to satisfy all age groups .  well , danger potential movie goers !  danger !  this movie crash lands without ever breaking thru the atmosphere of mediocrity .   & 0 & 0.816667 & 0.866667 & The overall sentiment of the explanations is negative, as they express disappointment with the movie's acting, dialogue, characters, script, and ending, emphasizing various shortcomings and mediocrity. The special effects are deemed as the only redeeming aspect. & 0.000000 \\
0.000000 & edward burns tackles his third picture with no looking back , and like his previous two , it is a working-class relationship picture .  however , unlike his previous work , the film dwells on a more personal story , and with a female protagonist .  and in no looking back , he stumbles , making a slow , boring film without the spark that enlivened his previous work .  claudia ( lauren holly ) is a small town waitress who is feeling stifled by her life .  she's at a turning point in her life , and feels as if she's going nowhere .  her boyfriend , michael ( jon bon jovi ) , is broke and in a dead end job .  if she were to marry him , she'd never get a chance to escape this town .  enter charlie ( edward burns ) , claudia's old flame .  he skipped town several years before , without any explanations . . . even  for claudia .  he has come back to town to see her , and suddenly she is torn .  should she stay with stable michael , and never escape her hometown . . . or  should she ignore her instincts and fall for charlie again .  part of the answer lies in the character of her mother ( blythe danner ) , who fell for the wrong man . . . and has spent her life pining for claudia's father to return .  now it seems that claudia is about to make the same mistakes .  at only a little past ninety minutes , no looking back is rather short for its genre .  unfortunately , it seems much much longer .  the storyline is simple and uninspired , and there's a lack of energy to the whole proceedings , which makes the entire drama rather tedious .  edward burns makes a misstep by casting himself in the crucial role as the egomaniacal old flame .  there's no one to restrain his ego , which reigns unchecked .  he walks into the room and lauren holly swoons . . . yeah ,  right .  lauren holly does what she can with her central character .  but we never understand why her character makes such pathetically bad decisions . . . and  we never really care .  bon jovi is the only sympathetic character in the whole movie .  his acting talents are much greater than they might seem , but he is given a mostly bland and ineffective role to work with .  there's not much to recommend in no looking back .  it's not that the film is bad . . . it's  simply boring .  there's no zest in any aspect of the film , and no reason to spend ninety minutes watching it .   & 0 & 0.866667 & 0.733333 & The film is described as slow, boring, uninspired, lacking energy, and tedious with a lack of character development and ineffective roles, leading to an overall negative sentiment. & 0.000000 \\
0.000000 & first impressions : critically , a close-to-awful film , but money-wise , it has been doing ( and will continue to do ) great .  a sometimes-funny film that sags and lags and oftentimes gets boring .  an orginal plot that grows old real fast .  one of the only 90 minute films that i've gotten bored through .  men in black has defied the odds .  when i first saw that the flick was 89 minutes long , i thought maybe that this was a poor attempt at an independence day type film that just ran out of gas .  however , i now realize that not only did men in black run out of gas , but the film in 90 minutes manages to show off a very original idea ( which summer audiences have embraced ) that becomes old about 25 minutes into the movie .  tommy lee jones and will smith play two " government " agents who are responsible for keeping order in alien society .  the ridiculous plot begins when an alien " bug " , played weirdly by vincent d'onofrio , who was so great in full metal jacket , lands on earth to retrieve a galaxy that's somewhere on " orion's belt . "  anyway , the basic plot revolves around jones and smith to stop this bug from getting the galaxy , or a higher power will blow up the earth .  the premise is ridulous , but that is not why i didn't like this film .  i love original plots .  this one had an original one .  but director barry sonnenfeld did something to this film that ruined its plot : he made the film drag and also put in unncessary elements in it that are found in romance films .  whenever i saw d'onofrio's bug stomp and eat people in the film , it got terribly boring after a while .  while smith's wise-cracks did fill in the gaps , it wasn't enough .  and also , i cannot believe the screenwriters elected to have a sub-plot where tommy lee jones missed his former lover because as an alien agent , they can't have contact with any humans really .  and here i see jones , at a satellite computer , watching his lover plant the garden ?  a sentimental moment in an alien movie ?  nice try , but i don't think so .  it doesn't work here .  it just makes the movie even more ridiculous and even more boring : we don't only have aliens to worry about , but now we have jones's conscience .  i came into the movie not wanting to see jones's conscience , but wanting to see a real action movie that had lots of aliens in it .  maybe it's unfair that i partly judged this movie on what my expectations were .  nevertheless , even though some parts are indeed funny , the plot in this movie grew old and boring -- quick .   & 0 & 0.800000 & 0.866667 & The overall sentiment of the text is negative, with the language used to describe the film being predominantly negative. The reviewer expresses disappointment with the plot, unnecessary elements, and the movie's tendency to become boring. Some positive aspects, like the original idea and funny parts, are mentioned but do not outweigh the negative tone. & 0.000000 \\
0.000000 & tim robbins and martin lawernce team up in this road movie comedy .  robbins plays an exec who discovers his wife having sex with his boss .  he goes into depression , and drives around his neighbourhood until he arrives inside the usual 'ghetto' side of every american city .  there , lawernce attempts to steal his car , but to no avail , and is dragged along with robbin's on a trip to arizona .  there , they hold up a store , are mistaken for two other robbers ( just like in my cousin vinny ) and are chased by the police , and the other robbers .  of course , there's gags along the way , usually from lawernce .  although the film is midly funny , and quite watchable , there's something so horribly familiar about it all .  this film should really be called beverly hills midnight run there's lawernce with his wisecracking and heavy profanity , just like eddie murphy in beverly hills cop , and pratically the same idea as midnight run .  it's full of all the road movie cliches , and even has a 'comedy' car chase , which just seemed so eighties .  even the credit sequence seemed to be out of a steve martin , or chevy chase , eighties comedy .  there also isn't much chemistry between robbins and lawernce .  in planes , trains and automobiles , there was real chemistry between steve martin and john candy .  in nothing to lose , there is hardly chemistry at all .  towards the end the chemistry seems to work , but for the bulk of the film , there is hardly any ,  the director directs the film in a workman like fashion , but gets the jokes across .  and although there is no chemistry between the two leads , they give good performances on their own .  lawerence gives a good eddie murphy performance , and robbin's is alright aswell .  however , robbin's performance is nowhere near as good as the characters he played in jacob's ladder , and the player .  in fact , if you want to see robbin's do comedy much better than in nothing to lose , i suggest you rent out the hudsucker proxy , where he is much funnier .  nothing to lose does have some funny moments in it , however .  the humour isn't particulary sophisticated , but if you enojyed dumb & dumber , you might enjoy this .  the characters in arizona are funny , and there's also a good scene where robbin's asks a shopkeeper which 'threatening approach' was better , lawerences ( which consisted of lawerence threatening to shoot the shopkeeper's ass , and swearing a lot ) , or himselfs , ( which was speaking in a deep , scary voice ) but the gags are predictable , the lack of chemistry infurating , and the ending too far-fetched and 'perfect'  nothing to lose then , is robbin's first 'bad' movie ( and hopefully his last ) , and suggests that odereick should twice before making another film .  he nearly ruined carrey's career with ace ventura : when nature calls , let's hope he hasn't ruined robbin's ( or even lawerence's ) with nothing to lose .   & 0 & 0.766667 & 0.833333 & The film "Nothing to Lose" receives negative reviews due to the lack of on-screen chemistry between the leads, predictable gags, unsatisfying ending, unfavorable comparison to other comedies, and disappointment in Tim Robbins' performance. & 0.000000 \\
0.000000 & well , here's a distasteful , thoroughly amateurish item that , surprisingly , was actually a box-office hit at the time of its release .  after just viewing the film for the first time , my primary question is how did anyone with an iq north of 35 enjoy this movie ?  it is cheap , idiotic , unfunny , and not nearly as raunchy as i had heard it was .  at least some smut would have livened things up a bit .   " porky's , " tells the story ( if you can call it that ) of four clueless high school buddies , pee wee ( dan monahan ) , billy ( mark herrier ) , tommy ( wyatt knight ) , and mickey ( roger wilson ) , whom desperately want to get laid .  women , for the most part , are a mystery to them ( and in this movie , they are to the audience , as well , since all of them are written and acted as if they are aliens from a different planet ) .  their plan is sidetracked , however , when they venture out to a smarmy strip bar named porky's , which they are able to get into using fake id's .  after they pay the manager one-hundred bucks for three hookers , they are played a trick on and find themselves being dumped into the swamp below the building .  for these four teenage guys , this means war on porky's .   " porky's , " ultimately manages to fail on almost every possible level .  as a teenage sex-romp , it is not wild or amusing enough .  as a comedy , all of the jokes are predictable and fall flat .  as a look back at the 1950's , it is something , i suspect , most people from that era would want to bury deep in a grave .  and as a revenge movie , it is a crushing bore .  one of the most offensive things about , " porky's , " is how jaw-droppingly inaccurate the film is about teenagers .  the four main characters are not even attempted to be developed as characters , and we learn very little about them , except that they are horny and would probably feel more comfortable in a preschool .  that sure is revealing information .  the female characters fare even worse under the inaudacious screenplay and direction by bob clark .  the women are all treated as objects or comedy props , rather than real people .  for example , the two gym teachers , mrs . balbricker ( nancy parsons ) and honeywell ( kim cattrall ) , only have one purpose , and that is to be made fun of .  mrs . balbricker is a gruff , no-holds-barred , overwight woman who will not stand for any foolishness , and , in one particularly embarrassing scene for all involved , honeywell has sex with a coach and barks like a dog .  there is no way to tell if parsons or cattrall are servicable actresses ( even though i have the suspicion they are not ) , but one thing is for sure : they are asked to do things in this film that are not at all funny , only humiliating .  to prove how out-of-touch this so-called comedy is , compare it to other more serious 80's films about teenagers , and it looks even worse in comaprison .  any of the john hughes pictures , such as 1984's " sixteen candles , " or , 1985's " the breakfast club , " put , " porky's , " at an even greater shame .  those films actually dealt with serious teen matters , but remained a great deal funnier , thanks to their bright and truthful writing .  and heck , if , " porky's , " wanted to be a teenage sex movie , i'd take 1982's " fast times at ridgemont high , " or 1982's " the last american virgin , " over this any day of the week .  director bob clark is not a bad director .  two years after he made , " porky's , " he directed the nostalgic holiday classic , " a christmas story . "  i would forgive him for this misfire , in fact , if it wasn't for the fact that he also wrote the screenplay .  just the thought that someone would actually sit down to write such a piece of garbage , and think that it was actually a film worth releasing unto the unsuspecting world , is actually a whole lot funnier than anything in , " porky's . "   & 0 & 0.866667 & 0.766667 & The text expresses a strongly negative sentiment towards the movie "Porky's" due to its distastefulness, lack of humor, and offensive portrayal of women. The reviewer also criticizes the undeveloped characters and compares it unfavorably to other 80's films about teenagers. & 0.000000 \\
0.000000 & a documentary from the twin hughes brothers , allen and albert ( dead presidents , menace ii society ) , about street pimps , all of whom are african-american .  an offscreen interviewer questions them about their lives and profession .  it is mostly the pimps who do the talking for their ho's , and the egotistical and flamboyant way you would expect them to be , is what you get .  i learned that the word " bitch " is their favorite word , as it seemed to crop up in every sentence they used .  it makes for a film with no surprises or nothing fresh to say , but if one has a sense a humor for what their lifestyle is about , then some of what they rap about might seem amusing .  they seem to all want to be thought of as businessmen , in the business because it is the easiest way for them to make big money .  it's also a power trip , accomplished by manipulating the girls to work for them , mostly by humiliating them and keeping them in place .  this sleazy pic , soon became grating and wore out its welcome to my unreceptive ears .  these verbose pimps had a smart answer for everything and never knew when to shut their face .  the hughes brothers used as their pimp role models , the feather-hatted , fur-coated , diamond ring-wearing , gold chain wearing , flashy cadillac-cruising pimp of the late '70s blaxploitation movies--like the mack and willie dynamite .  also used as reference , was iceberg slim's best-seller pimp , the story of my life .  we meet pimps such as : fillmore slim , c-note , charm , k-red , gorgeous dre , bishop don magic juan , and rosebudd .  they readily discuss their business arrangements : including percentages , lifestyles , knockin ( stealing another pimp's ho ) , and the thrill they get from women giving them money .  these dudes needed no prompting to talk , as they just love to brag about about themselves .   " priests need nuns , '' yaps c-note , a san francisco pimp .   " doctors need nurses .  so ho's need pimps . ''  as he tries to persuasively make the same case all the other pimps in the film make ; that the girls need them to show them the ropes , how to make money , to be their security blankets , protectors , and counselors .  the race issue was brought up right from the film's onset , as a number of white interviewees , regular citizens , note that their impression of a pimp , is that he's the lowest form of human .  while , we are told , in the black community , the pimp is looked upon as a successful entrepenaur , riding around in fancy cars , flashing wads of money , dressed in a flamboyant style , where his snake-gaiter shoes might cost him a grand , and where he boasts hollywood style status in the community .  we also see that the black pimps have a number of white girls in their stable and we see how they treat them like dirt , as the film implies that this could be payback for the days of slavery , of master and slave relations .  the film was all about pimp style and their projected image , as they run a hard-sell riff about the virtues of their work .  there are different styles of pimping , but the film mentions them only as being " macks " and " players " or " real pimps " and " perpetrator pimps , " but no further clarity is attempted .  it is also failed to get the women's side to this story of abuse , primarily taking the pimp at his word .  the film could never get to the core of what makes the pimp think in such a perverted way as he does , as the filmmakers were taking for a ride by the pimps , who were looking only to pose in front of the camera and say their thing , hungry for their 15-minutes of fame .  the only laughs i got out of this bleak look at an american subculture , was hearing how a few pimps retired and what they are now up to .  danny brown became a blues singer so he could keep his pimp's wardrobe .  while a hollywood pimp called rosebudd , married his ho and turned square , working to support his wife and daughter as a telemarketer .   & 0 & 0.866667 & 0.800000 & The overall sentiment of the explanations is that the text and documentary exhibit negative sentiment towards pimps, highlighting their exploitative behavior towards women and the filmmakers' failure to provide a balanced portrayal or explore the perspective of the women involved. & 0.000000 \\
0.000000 & hong kong cinema has been going through a bad spell .  the last few productions have been effect laded action adventures that combine both the best and worst of american filmmaking with the same qualities of hong kong films .  in a nutshell , the current crop of films from hong kong has been maddeningly convoluted and visually sumptuous .  with the one time british colony reverting back to mainland ownership , a lot of hong kong's best talents have crossed the pacific to work on u . s . productions .  such talents as jackie chan ( rush hour ) , chow yun-fat ( anna & the king , the corrupter ) and yuen woo-ping ( the matrix ) have all moved into the budget bloated world of hollywood filmmaking with mixed results .  now we can add two other hong kong filmmakers to the mix with star jet li and director and fight choreographer corey yuen kwai .  unfortunately " romeo must die " bears all the trademarks of a typical hollywood action film and none of hong kong's rhythms .  the film opens in a nightclub as an asian couple is necking .  enter a group of chinese gangsters led by kai sing ( russell wong ) .  kai confronts po sing ( jon kit lee ) , the son of kai's boss and leader of the local chinese family .  a battle breaks out between the bodyguards of the club and kai , who handily kicks and punches his opponents down .  it's not until club owner silk ( rapper dmx ) , bears down on kai and his henchmen that the fight ends .  the following morning po sing is found dead .  suspicions escalate , as issac o'day ( delroy lindo ) is told of the murder .  his concern that the war between his and the chinese family may explode and ruin his plans to move out of the business of corruption and into a legitimate venture .  issac implores his chief of security , mac ( issiah washington ) to watch after his son and daughter .  the scene shifts to a prison in china , where han sing ( jet li ) learns of his brothers murder .  he fights with the guards and is dragged off to be disciplined .  hung upside down by one foot , han recovers and battle his way out of custody in a blistering display of fight choreography and stunt work .  escaping to the u . s . han sets out to find the person responsible for his brother's death .   " romeo must die " is in many ways a fun film .  it is both absurd and assured .  the basic plot of a gangster wanting to become legitimate echoes " the godfather " .  the relationship between jet li's han and aaliyah's trish o'day reminds us of abel ferrera's " china girl " , except that romeo must die's couple never once exchange more than a loving glance towards one another .  their romance is much more puritanical than any other romance in film history .  the performances are adequate if not fully acceptable .  li , of course has the showiest part , having to express both an innocents and steadfast determination .  allayah , in her feature film debut manages to carry what little is asked of her with a certain style and grace .  it's obvious that the camera loves her and she is very photogenic .  but , still the part is under written in such a way that even a poor performance would not have affected it .  delro lindo as issac o'day carries himself well in the film .  an unsung and under appreciated actor , mr . lindo turns out the films best performance .  the other performers are all adequate in what the script asks of them except for d . b . woodside as issac's son , colin .  the performance is undirected , with the character changing his tone and demeanor in accordance with whatever location he is in .  an unfocused performance that should have been reigned in and / or better written .  first time director andrzej bartkowiak does a workmanlike job in handling the film .  having a career as one of the industry's best cinematographers , bartkiwiak knows how to set up his shots , and " romeo must die " does look good .  but the pacing of the film is lethargic , only coming to a semblance of life during the fight scenes .  the script by eric bernt and john jarrell is not focused in such a way that we can care about the characters or the situations they are in .  the big gambit of buying up waterfront property to facilitate the building of a sports center for a nfl team is needlessly confusing .  and of course the common practice of one character being the comic relief of the film becomes painfully obvious here as anthony anderson as allayah's bodyguard , maurice has no comic timing whatsoever .  the best things about the film are its fight scenes .  jet li is a master of these intricate physical battles .  one needs only to see his film " fist of legend " to understand that the man is without peer in the realm of martial art combat .  here , jet is given the opportunity to show off in a way that " lethal weapon 4 " ( jet's u . s . debut ) didn't allow .  unfortunately , a lot of jet's fights are aided with computer effects that detract from his ability and precision .  also " romeo must die " must be noted as having the most singularly useless effect ever committed to film , and that is an x-ray effect that appears three times during the course of the film , showing the effect of bone crushing blows on an opponent .  obviously a homage to the famed x-ray scene from sonny chiba's " streetfighter " , the scenes here are just pointless and interfere with the pacing of the film .  it's as if the film has stopped and a video game has been inserted .  one problem though about the fight scenes .  those that are familiar with hong kong action know that even though the films are fantasies and are as removed from reality as any anime or cartoon .  they do have an internal rhythm to them .  a heartbeat , so to speak in their choreography .  the fight scenes in a hong kong film breath with an emotional resonance .  this is created by the performance , the direction and the editing .  here in " romeo must die " , there is no staccato .  every fight scene , even though technically adroit and amazing becomes boring as the editing both cuts away from battle at hand and simple follows a set pattern .  the rhythm is monotonous .  a hong kong film has a tempo that changes , heightening its emotional impact .  'rmd' is limited to a standard 4/4 tempo , not allowing for any emotional content whatsoever .  a fine example of this difference can be found by examining a couple of jackie chan's films . .  watch the restaurant fight from the film " rush hour " and notice that the context of the fight , while technically amazing is rather flat ( the framing and cut always do not help ) .  now look at the warehouse fight from " rumble in the bronx " .  there you have a heartbeat , and emotional draw that doesn't let the audience catch its breath .  the stops and pauses for dramatic effect work perfectly , causing the viewer to be both astounded and flabbergasted .  here in 'romeo must die' , the fight scenes have no more emotional content or character than any john wayne barroom brawl .  jet li is a grand and personable screen presence .  it's a shame that his full talents were not used to full effect here .  one day filmmakers here in the u . s . will stop making films by the numbers and start to embrace the style and emotion that has made hong kong action pictures such a commodity .  until then , we'll be left with emotionally hollow product like " the replacement killer " and , currently " romeo must die " .   & 0 & 0.800000 & 0.833333 & The explanation highlights the negative sentiment towards the film industry in Hong Kong, particularly in regards to recent productions, with criticism towards "Romeo Must Die" for failing to capture the essence of Hong Kong cinema. & 0.000000 \\
0.000000 & the original babe gets my vote as the best family film since the princess bride , and it's sequel has been getting rave reviews from most internet critics , both siskel and ebert sighting it more than a month ago as one of the year's finest films .  so , naturally , when i entered the screening room that was to be showing the movie and there was nary another viewer to be found , this notion left me puzzled .  it is a rare thing for a children's movie to be praised this highly , so wouldn't you think that every parent in the entire city would be flocking with their kids to see this supposedly " magical " piece of work ?  a tad bewildered , but pleased to not have to worry about screaming kids and other disruptions that commonly go along with family films , i sat back for 97 minutes and watched intently and with a very open mind , having great expectations for the film .  looking back , i should have taken the hint and left right when i entered the theater .  believe me ; i wanted to like babe : pig in the city .  the plot seemed interesting enough ; after the events that took place in the original , babe the sheep-pig has become a legitimate national phenomenon .  but after a fateful encounter with a water well , arthur hogget ( james cromwell , who the movie could have used alot more of ) has been rendered bed-ridden for a number of weeks and the farm begins to go under financially .  the only solution that his wife ( magda szubanski , going from delightfully charming to downright annoying ) can come up with is to make an appearance with their new celebrity pet at a national fair ( i think ) and to use the money they earn to pay off the bank ( set aside for the moment the fact that they could get more than enough cash from donations if they just made their case known to the public ) .  problem is , the fair is being held in the middle of the dreaded " city " , a completely foreign place to both the pig and his companion .  setting the main plot in motion , mrs . hogget and babe travel to the unnamed city and shack up with a sweet lady who just happens to love to help animals , despite the law that you cannot keep them in hotels .  it is here that we meet an array of eccentric characters , the most memorable being the family of chimps led by steven wright .  here is where the film took a wrong turn .  up until this point , i had being having a rather enjoyable experience .  the beginning featured some smart writing and funny situations involving the farm animals from the first one and even an inspired moment at the airport where mrs . hogget is accused of smuggling drugs .  unfortunately , the story wears thin as we are introduced to a new set of animals that reside at the hotel , none of them being even one-tenth as interesting as the characters from the previous babe .  the main topic of discussion surrounding babe : pig in the city is the question of whether or not it is to dark and disturbing for small children , and i believe it is .  at one point , a dog is hung from his neck and slowly starts to drown .  at other times , we are treated to surrealistic flash-backs to mrs . hogget's full cavity search at the airport .  in fact , the overall tone of the movie is rather bleak and depressing .  however , that is , as they say , neither here nor there because kids will probably not like the movie anyway .  the animal characters and their plights were simply not intriguing enough to sustain my interest for an hour and a half , let alone entertain a child .  another problem i found with the film was it's sudden change of pacing and tone near the end of the story .  if you're going to make a darker and more sinister sequel , fine .  it may not be my cup of tea , but at least it is a noble ambition .  but to go from the downbeat feel of the rest of the movie and all of a sudden have slapstick finale with mrs . hogget swinging from wall-to-wall of a ballroom in elastic overalls ?  it just didn't feel right and was more painful to watch than it was funny or entertaining , and the same goes for the rest of the movie .   & 0 & 0.833333 & 0.833333 & The overall sentiment of the explanations is negative, as the reviewer expresses disappointment with the plot, characters, and tone of the movie, finding it uninteresting, depressing, and painful to watch. & 0.000000 \\
0.000000 & disney's " air bud " tells a boy-and-his-dog story with a twist -- the pooch is quite an accomplished basketball player .  granted , for a family comedy , it's not a very funny or successful idea to begin with , but it doesn't seem to matter -- " air bud " is surprisingly solemn .  save for occasional moments of forced slapstick , the movie wags its tale with a straight face -- not a very enjoyable approach .  if " air bud " had realized its own absurdity , then it possibly could have been better .  here , we're actually asked to cheer a moment when the dog marches out to save the big game , clad in two pairs of sneakers and even a jersey .  its number ?  k9 .  yeah , whatever .  the movie opens as golden retriever buddy ( as himself ) escapes from his current owner , abusive clown-for-hire norm snively ( michael jeter ) .  he ends up in fernwell , washington , where mopey new-kid-on-the-block josh ( kevin zegers ) is trying to cope with the move and the recent death of his father .  buddy , kevin .  kevin , buddy .  once the dog proves his on-court prowess , kevin's self-esteem rockets .  they both win places in the school's basketball team , with the animal as their mascot .  but before they can make it to the finals , snively surfaces to reclaim buddy .  everything plays out just as one would think : heavy on predictability , light on an actual story .  the movie follows a calculated chain of events -- kevin's gloom fades , snively gets his comeuppance and buddy contracts rabies and must be executed old yeller-style .  okay , that last one's a lie , but at least it would have been a quicker send-off than " air bud " 's courtroom climax -- no joke !  there's even a faux-cute musical montage where a reluctant buddy gets cleaned-up to " splish splash . "  paint cans are spilled and newspapers are buried , all in the name of formula .  the end credits note that " no special visual effects were used in the basketball sequences of this motion picture . "  that very well may be true , and although to see a dog sink shots is quite a sight , a movie cannot exist on feat alone .  but " air bud " doesn't seem to be interested in anything else .  a friend of mine insists he saw buddy a while back on a segment of david letterman's " stupid pet tricks . "  i can't think of a more appropriate connection -- this movie is a stupid pet trick .   & 0 & 0.833333 & 0.866667 & The explanations provided present a mixed sentiment towards the movie "Air Bud", with some highlighting negative aspects such as lack of humor, predictability, and lack of substance or story, while others mention positive elements such as the dog's basketball abilities and the protagonist's self-esteem improvement. Overall, the sentiment leans slightly towards negative. & 0.000000 \\
\bottomrule
\end{tabular}
