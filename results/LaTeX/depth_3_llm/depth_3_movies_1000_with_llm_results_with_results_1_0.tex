\begin{tabular}{rlrrrlr}
\toprule
sentiment & sentence & sentiment_score & confidence_rating & explanation_score & explanation & results \\
\midrule
1.000000 &  ( dimension films , " scream 2 " 's distributor , has asked press to say extremely little -- if anything -- about the film's twisty plot .  that's no easy task considering the wit that deserves to be mentioned here , but i will do my best to be vague ( now , there's a first ) . )   " the first one was [good] , but all the rest sucked , " said a cinematically-savvy teen in last winter's wes craven thriller " scream , " her statement referring to the films of the " nightmare on elm street " series but really putting down franchise overkill in general .  the comment certainly carries clout : for every truly great sequel , there appears to be a couple of duds , making one wonder if writers are better off sticking solely with fresh ideas .  but like it or not , along comes " scream 2 . " and believe it or not , it's a doozy -- a slick , sinister , madly subversive good time at the movies , as intent on sending up hollywood's sequel syndrome as much as its prequel poked fun at slasher conventions .   " scream 2 " is definitely that rare movie thing -- a follow-up that can stand along side its original with pride .  it's been two years since a pair of overzealous horror movie fans clad in edvard munch-esque get-ups carved their way through the young populace of woodsboro , california .  those surviving the ordeal have gotten on with their lives .  plucky heroine sidney prescott ( neve campbell ) is a drama student at the midwestern windsor college ; her pop culture-whiz pal randy meeks ( jamie kennedy ) has tagged along .  trash tabloid reporter gale weathers ( courteney cox ) has written a best-seller based on their ordeal , the basis of which has been turned into a ( very bad ) movie called " stab . "  and dewey riley ( david arquette ) , still suffering from wounds inflicted during " scream , " has left his job as a police officer for a while .  life is tranquil . . .  at least for a while .  several sudden murders bring sidney , randy , gale and dewey together again , but with suspicious eyes cast on each other and most of those in their surroundings -- if these four people learned anything from the past , it's to trust no one .  thus , the possible victim/potential killer list includes said quartet , as well as : derek ( jerry o'connell ) , sidney's new beau ; cici ( sara michelle gellar ) , a chatty sorority gal ; joel ( duane martin ) , gale's cameraman who's not too thrilled with her blood-soaked past ; hallie ( elise neal ) , sidney's sassy roommate ; debbie ( laurie metcalf ) , a local reporter who gives gale some not-too-friendly competition ; and mickey ( timothy olyphant ) , randy's good friend and fellow film student .  cotton weary ( liev schreiber ) , the man sidney wrongly accused of her mother's murder in scream , also shows up on campus -- but why ?  like the first scream , craven and screenwriter kevin williamson inaugurate things with a bang .  this time , it's an extended sequence that finds an african-american couple ( jada pinkett and omar epps ) attending a sneak preview of " stab , " making sly references about everything from sandra bullock to the lunchmeat non-roles of blacks in slasher cinema .  before tragedy ensues at the screening , the movie-within-a-movie setup allows for plenty of jokey moments , including a scene from " stab " which turns scream's now-famous drew barrymore prologue on its ear .   ( " b---- , hang up the phone and star-69 his a-- ! "  pinkett yells at the screen . )  it's funny , creepy stuff ( i don't think i'll ever feel safe in a movie theater again ) that effectively foreshadows the ratio of smart scares and spoofy laughs down the road .  another flawless example of this is a riotous cell phone-assault of the killer on randy , dewey and gale in broad daylight ; it builds comedically and ends with genuine terror because of the emotional investment made on these delicious characters .  there are other superb set pieces , but explaining them risks ruining their effect .  there are more players this time around as to provide both a higher number suspects and a staggering body count .  it's something of a trade-in -- the new characters are too large in number to be as fully-developed as the original " scream " gang , but this movie packs in knowing performances ( especially by kennedy and cox ) and more death for your dollar .  it can be argued that , while the original might outclass it by a tad , scream 2 is both scarier and funnier .  it certainly doesn't skimp with shocks -- a major one being the killing of one cast member very near and dear to my heart ( i actually questioned craven and williamson's judgment here , considering how much this person has added to these films ) .  sometimes you wish that more could have been done with stab -- the two scenes we're shown are dead-on -- but for film fans , there are plenty of subtle nods to movies like " the usual suspects , " " aliens " and " the empire strikes back . "  still , " scream 2 " does so many things right , it's petty to quibble .  its penchant for parody is irresistible , discussions of the merits of film sequels and all .  the attractive , solid cast is a definite bonus .  and the denouement ( while not as shocking as " scream " 's ) is priceless ; the film's final revelation incorporates bits of a particular horror film prominently referred to in " scream " with one of the best bad guy motives ever put on paper .  all this and a " scream 3 " is in the works .  knowing the series' tendency for self-reference , perhaps one of its characters will again discuss the concept of sequel justice .  mentioning " scream 2 " 'll help them in their argument .   & 0 & 0.833333 & 0.833333 & The text provides a generally positive review of the movie "Scream 2," praising its slick and sinister qualities and comparing it favorably to the original. However, there are also mentions of disappointment and criticism towards the franchise's sequels. Overall, the sentiment leans towards positivity with some reservations. & 0.888061 \\
1.000000 & it may seem weird to begin a film about glam rock with a sequence that includes a spaceship , a green ovular pin , and the birth of oscar wilde , but if one really strains , they can see that perhaps maybe these connections are not half off .  wilde's philosophy was that everyone should be true to their own human nature , and the result of his following this philosophy was that he was imprisoned , loosing his family and his career .  the glam rock movement in the early 70s in england had a similar take and came to a similar demise .  the main difference was by placing makeup on their face and acting out on their deepest fantasies and inquiries about life ( mostly dealing with androgony and sexuality ) , they became less and less like themselves and more and more like everyone else .  and that's why the movement seemed to end as soon as it began .  it's been said that todd haynes' " velvet goldmine , " the film that chronicles what it was like to be a part of the movement , not only from those who experienced it but by those who created it , steals a lot from " citizen kane , " and that's true .  in the film , a british journalist in the 80s , arthur stuart ( christian bale ) , is asked to go back and find out what happened to 70s glam rock star brian slade , a fictitious rock star , who faked his own death on stage , bringing the end not only to his career but to the entire glam rock movement .  structure-wise , this totally steals from " kane , " not only from the set-up ( he interviews three people , and the story is a result of their flashbacks ) , but in other things , like the beginning ( death , then newsreel ) , and smaller details , like the bitter ex-friend in a wheelchair and the bitter ex-wife as a washed-up lounge singer , found in a bar after hours .  this is no " kane , " and it really doesn't aspire to be : it doesn't attempt to be the deep outlook of something gone like " kane " did , and it doesn't really uncover anything poignant about humanity .  instead , it uses the flashbacks as a form of contrast between the magic that was the glam rock era and the boredom that was life after glam rock for those who were participants .  the 80s scenes are dry and deliriously melancholy , equipped with a performance by bale that is perhaps appropriately dull and unengaging .  but the 70s scenes are engaging , though , not to mention addictively campy .  they radiate with gorgeous cinematography that nicely accentuates all the vibrant colors of the era , and a feel that's so eerily lighthearted that when combined with the then-footage , they become not only a symbol of decadence , but of times when everything seemed so simple .  instead of creating the world the way it was , haynes paints their own world as if it were a narcotic fairy tale : the glam rock movement was full of so much freedom and liberation that after you've experienced it , everything else seems so mundane .  there's no outside world to speak of , secluding these people inside a protective globe that will eventually crack .  haynes focuses his story on the tale of blade and the other fictitious rock hero , curt wild , and their relationship that created , molded , and then brought down the movement , reducing everyone else who contributed to it as merely that - contributors .  blade - played with reserve by the pouty jonathan rhys meyers - and wild - played with anarchic wildness by ewan mcgregor - are little more than thinly disguised recreations of david bowie and iggy pop , respectively , complete with the creation of a ziggy stardust persona named maxwell demon and the confirmation that the two rock stars may have very well not only have been one-time partners in music , but also partners in bed .  through discussions with slade's ex-manager cecil ( michael feast ) , ex-wife mandy ( toni collette , reinventing herself as an american blonde dish who fakes a british accent when with slade ) , and finally ex-partner wild , journalist stuart begins to remember his own experiences in the era , like his discovery of his rebelliousness cum conformity , including the moment when he began questioning his own sexuality ( when he opens up his first slade record , he finds a naked and green slade lying on a crimson blanket ) , and finally running away from home to be part of the london scene , finally resulting in leading a boring job in america .  haynes demonstrates that he's quite the visual auteur , molding scenes that are like long heald breaths , such as a seemingly long sequence juxtaposing a slade/wild concert of them performing a brian eno cover , " baby's on fire , " with scenes from a decadent drug party ; and the film's most wow-inspiring sequence , the first concert scene of wild with his band , the ratttz , where mcgregor lets loose so much anarchic steam that his wild iggy-esuqe movements ( including stripping naked ) and screams that the film captures that perfect moment when one discovers a major talent , and another ( slade ) discovers his idle .  even the brief music videos , spoofs of bowie's , have a rare visual flair that's pure camp , and which would cause ken russell to drool .  the best sequence , though , may be the beginning , following the prelude , a sequence which acts as the middle ground for both the actual being of the movement and the post-movement era .  in it , stuart and his mates are going to the infamous slade concert where he fakes his own death , where he kills off his alter ego maxwell demon in what appears to be a real assasination , which brought about the end of the era in one swift fake bullet .  with brian eno's famous " needle in the camel's eye " playing in the background , the scene has a detached exhileration - the song doesn't seem to be played in quite the same way the other songs are .  it has a distance that's hard to put a finger on , and it seems to represent that all this is coming to a quick and sad ending , and when one sees slade in the dressing room before the show , docked in a silver frock with wings and blue hair , depressingly staring into the mirror , it comes off as a none-too-obvious prophecy of the finale in the beginning of the film .  this movie's not so much about plot , but more about the way it is presented , making this one of those films which is classified as being " style over substance , " a statement which prompts many critics to line up for attack .  however , for the most part , the engaging part of this film is not the story but rather the way in which haynes creates this world , by using his sets , costumes , cinematography , and especially the music to play as characters in his film .  the music is especially notable .  the soundtrack , which is wall to wall , consists of old school glam rock tunes by the likes of brian eno , t-rex , and roxy music , as well as covers by slade's band ( with vocals by thom yorke , and occasionally rhys meyers himself ) , and even newer music by shudder to think that sounds uncannily bowie-like .  the cast is rather impressive , but no one really walks away with the film , and no performances are extremely good .  although eddie izzard , as slade's manager ( who challenges slade's first manager to an arm-wrestling match to see who gets control of his career ) and michael feast as the first and tragic manager come off greatly , and collette and mcgregor have their moments , rhys meyers and bale are noticably sub-par , neither putting a lot of effort into their respective roles .  some of it is at fault with the actual construction of the film by haynes .  rhys meyers' brian slade remains merely a metaphor for the glam rock era , dying when he turned into a retro-garbo , resorting to a life of salinger-ism , and found in the mid-70s to be lying around , sniffing coke off the ass of a party girl .  in fact , no one in this film is really seen as a person , rather than just as a symbol or composite of a type who thrived during the era .  bale's character is an especially tough sell : bitter and depressed by flashbacks to his young adulthood , he's not an extremely personal character , and his cliched experiences ( being hounded by the record store guys for buying a record put out by a " poof " ) never help us communicate with him .  nevertheless , rhys meyers and mcgregor have the excuses that their characters are not really characters but rather the results of an era that has left them , and others , bitter , part of which helps the film work , since this is a major piece of eye candy .  at the prime of the movie , they at least look the role of fashionable leaders - slade with his perpetually-changing , androgonys persona , and wild with his topless and unpredictable image , which clash and fuse into an unstable union .  their story - of how slade was so influenced by wild that he adapted it into a similar who-cares attitude crossed with camp - is engaging , without the personal background .  the story of stuart , though , is a tougher sell because he's the everyman , and when you can't totally identify with the everyman , your story's in slight trouble .  as a cultural rock piece , " velvet goldmine " is rather good , but it is merely good .  it never totally takes off , although it has moments where it absolutely flies , but then comes back down .  it's really nothing more than a bunch of really great moments , surrounded by material that could really be much better .  the beginning is captivating but slow , the middle is fantastic , and the ending is not only shallow but worse than that , it drags ( the final half hour not only does it not bring the story to any real conclusion , it could probably stop at any point ) .  and did anyone really figure out what the mystery is that arthur unearths ?  it almost seems like there was no point in the entire investigation other than to unearth the past , which is commendable but not totally so .  still , it does what the oliver stone pic " the doors " didn't do right , becoming very insightful to what happened , using the visual style to hit most of the right notes of contrast between what made the era so great to those who lived during it and why after living through such an era that everything else seems half-assed ( i suppose the message is : " live life , but after you've lived it , what else is the point when you'll be forced to live with regretful memories ? " ) .  the really superb achievement of this film is that during its flashbacks , it successfully creates for the audience what it must have felt like to be in that era , but with the added perk of knowing the result of all that goes down .   & 0 & 0.769600 & 0.800000 & The overall sentiment of the explanations is slightly negative, as they highlight the downfall of the glam rock movement, criticize the construction of the film, and mention sub-par performances. However, there are also mentions of engaging moments and the successful portrayal of the glam rock era. & 0.834711 \\
1.000000 & in may of 1977 , just 2 years after steven spielberg's success with jaws and 3 years after francis coppolas' the godfather , a risky , ambitious young director named george lucas went for the same unpredictable box office success .  little did he know that star wars would become the greatest science fiction epic in the history of film .  in january of this year , lucas re-released star wars on its 20th anniversary .  in this new , remastered version , the added effects , which range from leathery desert beasts inserted into already existing shots to an awkward new scene in which hans solo bargains his way out of a jam with a computer-generated jabba the hutt , don't do much but call attention to themselves .  other than these , the film is exactly the same .  why pay to see it in the theaters , you might ask ?  simply because when this movie was released , most of us were a few years short of being born .  we should all get up and go to the movies for the experience , especially if you haven't seen it .  the plot is basic but in the same way , complex .  it has many themes , but the one rotates around luke skywalker ( mark hamill ) and his quest to become a jedi knight .  his mission begins when a droid named r2-d2 plays back a secret message recorded by the beautiful princess leia ( carrie fisher ) , which tells him to contact the mysterious obi-wan kanoby ( alec guiness ) .  he and kanoby then , joined by two inter-galactic renegades , hans solo ( harrison ford ) and chewbacca , and skywalker's two droids d2 and c3po ( the voice of anthony edwards ) , wage war upon the corrupt empire , commanded by an evil general ( peter cushing ) and the traitorous darth vader ( the voice of james earl jones ) .  skywalker defeats the evil empire by rising out of himself to embrace something larger : the force .  the film is obviously quite star-studded .  but how was lucas to know that ford would become one of the most sought after men in hollywood and jones' voice would appear so many more times in the future ?  it just happened to work out for him .  the magic of star wars lies in the way that skywalker's triumph is symbolized by the audience's sense of becoming something larger than life itself-a universe of fans , young and old , recreating a movie atmosphere that will impress generations forever .   ,   & 0 & 0.866667 & 0.800000 & The overall sentiment of the explanations is slightly positive, emphasizing the impact and success of Star Wars. However, the mention of unnecessary added effects and questioning the value of seeing it in theaters slightly lowers the sentiment. Overall, the positive elements outweigh the negative ones, resulting in a slightly positive sentiment. & 0.991622 \\
1.000000 &  " a bug's life " may not be " toy story , " but it's more close than " antz " was .  i really liked " antz , " basically because , yes , it is clever and witty and intelligent ( and it has the temerity to take a chance and put woody allen in the lead , perhaps the year's most inspired casting ) , but there was , in fact , something missing from it , and i'm the first to admit that .  maybe it's that it never totally lets go and takes off into real innocent fun - it's too obsessed with its orwellian message to become totally engaging , and if it weren't for woody allen , it would have been just a really clever good-not-great flick .   " a bug's life " has a similar premise , and it also has disney to insure that it's g-rated and not totally over kids' heads like " antz " was ( not a bad thing , believe me ) , but what it also has is a tone that's completely innocent even when it's also remaining perpetually clever .   " antz " is still the more witty film , and i love it for it , and " a bug's life " is more for general consumption , but it's also more entertaining .  it broadens its horizons , and when it really moves from the ant colony , it really shows us a whole new world we've never seen before , and it's take on the evolution of bugs is a lot better than the one in " antz . "  it's also light and clever .  the visuals are , instead of earth tones , bright worldly colors , and still gives the amazing visual technology of " antz " at least a worthy contender .  the characters are also nicely realized - though watered-down for broad consumption , they still carry more wit than the contemporary disney animated films .  the lead character , flik ( voiced by dave foley ) , is nothing more than a slightly-less neurotic z from " antz , " but foley makes him nearly as engaging as allen made z . instead of gene hackman playing the villain , we get the much more menacing kevin spacey as the lead grasshopper , the nicely-titled hopper .  and for the romantic lead , we don't get spacey sharon stone , but intelligent and hard-to-get julia-louis dreyfuss , who doesn't even come around until the very final frames .  not that the plot's really any better than the one in " antz " - it's basically yet another redux of " the seven samurai , " with an ant colony under the control of giant grasshoppers forcing them to produce a product for them or else .  when flik , a bone fide inventor , creates a time-conserving apparatus that accidentally destroys the season's donation , he puts them all in risk , and is sent away so that he won't screw anything up with the pretense that he is searching for help to fight the grasshoppers .  he runs into a group of " warrirors " who are , unbeknownst to him , a group of carnival bugs , and they agree to help under similar false pretenses .  these bugs are an equally wonderful assortment to anything in " toy story " : foppish walking stick slim ( david hyde-pierce ) , german caterpillar heimlich ( joe ranft ) , quick-tempered and insecure male ladybug francis ( denis leary ) , pretentious praying mantis manny ( jonathan harris ) , his assistant butterfly hypsy ( madeliene kahn ) , spider rosie ( bonnie hunt ) , and two fleas , tuck and roll ( michael mcshane ) , who speak in undiscernable jibberish .  the writers lightly touch on each bug's place in bug society and the malleability thereof while making a wisecracks at everything they can , and blowing the audience away with wild visual treats .  grasshoppers jumping in unison seems like a menacing earthquake .  a small bird becomes an ominous mortal threat , whose usual mild-mannered squeal is a scream of death .  the bug city is a modern-day metropolis , complete with fireflies temping as traffic lights and a fly sitting on the curb , holding out a cup , with a sign lying next to him that says " kid tore off wings . "  and a rainstorm is like a giant flood , with each drop acting like a small bomb dropped at millions of miles per hour .  around this , the pixar animators stage several large set pieces , like a resuce mission halfway through that is as wild and entertaining as anything this year , and a wonderfully exciting action piece at the end , a chase scene at night through the labyrinthine branches of a small thicket .  meanwhile , each character gets the spotlight to be completely idisyncratic and interesting , something " antz " couldn't do , and by the end , over the end credits , " a bug's life " pulls it's final punch , out clever-ing " antz " with a series of incessantly hilarious faux-bloopers that come just at the right time , when those who leave immeadiately after the final frame of a film have left , and you can brag that you were one of the elite who stayed and got the full money's worth of entertainment .  still , it's no " toy story . "  that film , maybe above any animated film i've ever seen , encompassed almost true perfection in story , character , and wit .  it proved that it didn't need to brag about it's cutting edge technology to really soar ( it's the least visually striking of the three computer animated films thus far , but is still the most satisfying ) , and created a perfect world of idiosyncratic delights and innocent fun .  yet " a bug's life , " as well as " antz , " are still amazing films , and prove without doubt , that if you're gonna make a computer animated film , send it to the guys who created these three flicks , evne if they're not working under the same roof .   & 0 & 0.800000 & 0.833333 & The text presents a slightly negative sentiment towards both "A Bug's Life" and "Antz", but leans more towards positivity for "A Bug's Life". The author acknowledges the cleverness of "Antz" but mentions something missing from it. They describe "A Bug's Life" as innocent, entertaining, and visually appealing, with engaging characters and comedic moments. However, they believe "Toy Story" is a superior animated film. & 0.957599 \\
1.000000 & this summer , one of the most racially charged novels in john grisham's series , a time to kill , was made into a major motion picture .  on january 3 of this year , director rob reiner basically re-released the film under the title of ghosts of mississippi .  based on the true story of 1963 civil rights leader medgar evars' assassination , ghosts of mississippi revolves around the 25-year legal battle faced by myrlie evars ( whoopi goldberg , sister act ) and her quest to have her husband's obvious assassin and racist byron de la beckwith ( james woods , casino ) jailed .  so she turns to assistant district attorney and prosecutor bobby delaughter ( alec baldwin , heaven's prisoners ) to imprison the former kkk member .  ghosts sets its tone with an opening montage of images from african-american history , from slave-ship miseries to life in the racist south of the 1960's .  but all too soon , the white folks take over , intoning lines like " what's america got to do with anything ?  this is mississippi ! "  as beckwith , james woods , with his head larded with latex most of the time as an old man , teeters between portraying evil and its character .  meanwhile , goldberg turns in a very serious and weepy performance as the wife who wouldn't let her husband's death rest until she got the conviction .  both deserve serious oscar-consideration .  this brings us to the dull performance of baldwin .  let's face it , trying to match matthew mcconaughey's wonderful acting in a time to kill is basically impossible .  and baldwin is living proof of this , as no emotions could be felt .  it seemed as if he actually had to struggle to shed a single tear .  either poor acting or poor directing , but something definitely went wrong .  another strange mishap was the fact that goldberg's facial features didn't change , as she looked the same in the courtroom as she did holding her husband's dead body 25-years earlier .  yet woods' was plastered with enough make up to make him look like goldberg's father .  at least the make-up was realistic .  with some emotional moments in the poorly written script , ghosts of mississippi lacked in heart , when its predecessor , a time to kill , brought tears to everyone's eyes .  don't get me wrong , the movie wasn't all that bad , but if you've seen grisham's masterpiece , then don't expect this one to be an excellent film .   ,   & 0 & 0.766667 & 0.766667 & The overall sentiment of the explanations is slightly negative, with criticism towards dull performances, poor acting/directing, and a poorly written script, although some positives were noted, such as realistic makeup. & 0.980323 \\
1.000000 & not since attending an ingmar bergman retrospective a few years ago have i seen a film as uncompromising in its portrayal of emotional truth as secrets & lies .  like bergman , director mike leigh is interested in probing his characters' inner depths through hypernaturally blunt confrontations .  also like bergman , leigh engages in frequent closeups of his characters' ravished and wracked faces .  and the prominent mournfulness of a cello on the soundtrack recalls bergman's own use of a bach cello suite in an earlier film .  all that is missing is a discussion of god .  which is not to say that secrets & lies is nothing more than an homage to the swedish master .  in fact , it is quite possible leigh had no such intentions in mind .  nonetheless , what we get is so far removed from the average moviegoing experience -- even from the reason we go to the movies in the first place -- that it takes some effort to adjust to the film's rhythms .  once the adjustment is made , however , there are great rewards .  one such is the chance to see life on the screen as it really is .  though leigh may have adopted some of bergman's stylistic touches , most obviously in an early scene of terse cross-cutting during a married couple's strained conversation , as well as in that somewhat obtrusive score , the overall feeling of the film is that it eschews any " style " at all .  whereas bergman uses artifice as a tool to expose reality , leigh makes the camera a mere observer , almost as in a pbs documentary .  the effect of this is to focus all of your attention on the actors .  it is a tribute to everyone involved that , despite such scrutiny , only infrequently are we aware that anyone * is * acting .  much has been made of brenda blethyn's performance , and rightly so , but it is only when you remind yourself that you are watching a fiction that you realize how good she is .  there are a few missteps .  for one , except for one scene ( tragicomic , as it happens ) , there is scant humor in the film .  this leads to a certain monotonous tone throughout .  and occasionally ( as with bergman ) the bluntness of the situations can seem forced .  for all that , this longish film manages to keep hold of your attention .  it is unfortunate that the audience for secrets & lies will most likely be limited to an intellectual elite , for there is nothing inherently intellectual about this film .  in fact , it might easily resonate more strongly for millions of working class filmgoers who will likely never see it .  there is even a sweet but significant irony in the film's unspoken take on race relations , something an american audience at least would do well to observe .  nonetheless , secrets & lies is not for the faint of heart .  though there is nothing physically horrific to make one squeamish , the exploration of common human frailty can be so raw and unsparing that it is tempting to turn from the screen .  needless to say , it is also very depressing at times .  but for many of us , of course , so is life .  and though the film is too honest to tack on a phony happy ending , that same honesty allows it to admit that things can also get better .   & 0 & 0.833333 & 0.900000 & The film is praised for its uncompromising portrayal of emotional truth and the rewards of seeing life as it really is, with strong performances and resonance with working-class audiences. However, there are some missteps, a monotonous tone, and occasional forced situations that contribute to a slightly negative sentiment. & 0.019231 \\
1.000000 & 1992's alien3 marked not only the death ( by suicide ) of its popular protagonist , ellen ripley ( sigourney weaver ) , but , in many ways , the alien franchise itself--box office receipts were anemic , thanks to poor audience word of mouth ; and the critics who rallied around the first two installments , 1979's alien and 1986's aliens , savaged david fincher's slog of a sendoff ( myself included ) .  hence , weaver , director jean-pierre jeunet , and the others behind alien resurrection faced a two-fold challenge--not only somehow resurrect ripley , but also rescue this once-profitable series from the scrap heap .  despite the odds , they have succeeded , even if the entertaining new installment does not measure up to the excellent first two .  writer joss whedon devises a quick , easy , and painless answer to the dead ripley problem--clone her , which is what shady military scientists do using some blood left behind on fiorina 161 , the prison planet of the third film .  that done , the _real_ challenge presents itself--what do with her .  alien introduced ripley as smart and resourceful ; aliens simultaneously toughened her up and made her more vulnerable , exploring her maternal side ; alien3 saw her undergoing the seven stages of death .  what could be next ?  whedon comes up with a clever spin : since the original ripley died while impregnated with an alien queen , the blood used for the clone is also " infected " with alien dna .  so the new ripley is , indeed , new--a human/alien hybrid blessed with heightened instincts and strength , a psychic bond with the deadly species , and a more predatory attitude .  unfortunately , that is where alien resurrection's clever streak in writing stops .  the alien series is known for having stronger stories than most creature features .  but the story in resurrection is more of an afterthought .  the movie begins with a plot involving some military types attempting to train aliens to do their bidding , but once the creatures break free , it is once again ripley and a ragtag crew ( this time a bunch of interstellar smugglers , including tough waif call , played by a game winona ryder ) trying to exterminate them .  and the alien ripley scenario is ultimately not exploited to its full potential ; i would have liked deeper exploration into the quandary of becoming one of the species she has spent her entire life trying to destroy .  while the settling into tried-and-true formula is a little disconcerting , the formula is tried-and-true for a reason , and jeunet tackles the proceedings with giddy abandon .  the alien , after all these years , is still terrifying , and a new breed that is introduced is no less so .  the violence is appropriately grisly and extreme , and the action set pieces are suspenseful and exciting , most notably an extended underwater sequence .  the film is absolutely mesmerizing visually , thanks to the solid work done by production designer nigel phelps and cinematographer darius khondji .  as technically adept as jeunet's direction is , perhaps his ( and , for that matter , whedon's ) greatest contribution is the infusion of humor into this notably downbeat and serious series .  a sense of humor may seem to go against everything this horror show stands for , but the self-awareness of the excess just adds to the fun .  no , alien resurrection is not the great film that ridley scott's alien or the even greater film that james cameron's aliens was .  but after the dauntingly slow gloom and doom of fincher's alien3 , jeunet's resurrection is a welcome return to its roots as a wild , reckless thrill ride .  that is what made the alien series so popular in the first place , and that is what will keep the series popular in any future installments .   & 0 & 0.816667 & 0.816667 & The overall sentiment of the explanations is slightly positive, as they highlight the entertaining aspects of the film and praise certain elements like the successful resurrection of Ripley and the technical aspects. However, there is also mention of the story being an afterthought and not fully exploited, as well as acknowledgment that it is not on par with previous films in the series. & 0.997998 \\
1.000000 & for the first reel of girls town , you just can't get over lili taylor .  is it possible to remain unperturbed that a woman of this age is playing a high school student ?  harder still , can you avoid astonishment at how perfect she is in the part ?  with the sure , showy expertise of a de niro , taylor imbues unwed mother patti with all the faux fuck-you confidence she can muster .  it's a deeply felt , dead-on performance , and by the halfway point you've forgiven taylor her own maturity .  for all its verite pretenses , girls town is less about a bunch of high-schoolers than it is about a movie director and a cadre of writer-performers examining their own feelings about rape , relationships , and american womanhood .  the movie's opening scenes sketch an easy friendship between a group of four smart/sassy high school outsiders on the verge of graduation : patti , angela ( bruklin harris ) , emma ( anna grace ) , and nikki ( aunjanue ellis ) .  but about 10 minutes into the film , we get word that nikki has swallowed a fistful of pills and bowed out of life .  the remaining girls manage to smug nikki's diary out of her grieving mother's home , and page through it at patti's place , investigating the root of nikki's despair .  what they find is cruel and galvanizing -- nikki had been raped by an editor at a newspaper where she interned , and was privately questioning the sanity and value of a world where such a thing could happen .  shell-shocked by the truth of the matter , the three girls begin to talk among themselves and come to a realization about their own lives .  emma says she was raped last year on a date with a football player .  patti's barely sympathetic , wondering what the hell emma thought she was doing with her shirt off in a jock's car .  and anyway , patti's had her own share of guys who wouldn't take no for an answer .  pained , acrimonious debate ensues as the trio grow more and more angry -- angry about the impossibility of saying no , and angry with themselves for being weak , for being naive , for putting up with it .   " why do we put up with it ? "  emma finally asks , once the argument has reached a fever pitch .   " we try to talk about it , and look what happens .  we fight for 20 minutes . "  there's the admirable crux of this picture .  girls town is a political film , in that it calls for action .  but it's incomplete-it's never quite sure how to justify that action , or what form it should take .  the trio become crusaders for their right to dignity , and it's surely one of this year's most stirring movie moments when emma lobs a concrete block through the passenger-side window of that damn jock's car ( doing the right thing ? ) .  they hang out in the bathroom , chatting incessantly and scrawling slogans and naming names on the door to a toilet stall .   " subvert the patriarchy , " emma writes , and then starts a hall of shame list underneath that other students begin to contribute to .  the girls silence their hecklers , take revenge on the deadbeat father of patti's child , and eventually come face-to-face with nikki's tormentor .  still , the movie has a desperately unfinished quality , and none of these subsequent exploits is particularly satisfying .   ( you keep waiting for someone to call the cops on these brazen girls , but nobody thinks of it . )  perhaps searching for a resolution missing from the material at hand , the film borrows its epigraphs from audre lorde and queen latifah ( " who you callin' a bitch ? " ) .  the points are well taken , but they should have been unnecessary .  long on characterization but a little short on story , girls town is less than it could have been , and too self-congratulatory for my own taste .  director jim mckay is best known as a consort of r . e . m . 's  michael stipe ( he directed the nearly unwatchable tourfilm for that band ) , and girls town is his first foray into feature filmmaking .  the movie was created and scripted by committee , but mckay and his lead performers had to shoot on a shoestring when , predictably , the group couldn't secure funding .  then again , girls town on a budget would hardly be the same film .  the movie's slapdash quality is key to its significance .  truth be told , all three of the leads look too wise for high school , but that's ok -- there's a once-removed quality to their fine , spontaneous performances , as the women take a very skillful look back at the girls they once were , the girls they wish they had been , or some combination .  for all its flaws , this one gets extra points for having its uncertain head and heart in the same noble place .   & 0 & 0.843333 & 0.916667 & The sentiment analysis of the explanations suggests a mixed response to the film, with some praising the performances and thought-provoking themes, while others criticize the flaws and self-congratulatory nature. However, there is an overall positive sentiment towards the movie, highlighted by the praise for Lili Taylor's performance and the exploration of important themes. & 0.992791 \\
1.000000 & based on the boris karloff's classic by the same name , the mummy starts off with the high-priest of osiris , imhotep , who murders the pharoah for his mistress and is punished by being mummified in the most horrifying way possible -- bandaged up , having his tongue removed , and being covered in flesh-eating scarab beetles , then entombed . . . all while still alive .  recap a few thousand years later , where a soldier named rick ( played by brandan fraser ) aids a young historian named evelyn ( played by rachel wiesz ) and her brother , jonathan ( played by john hannah ) in finding the book of amon ra , in the process inadvertently freeing the mummy .  problem is that the mummy wants to revive his mistress , using evelyn as a sacrifice . . .  walking in with relatively low expectations , i thought this movie was actually pretty good .  the visuals and cgi are astounding , and obviously not cheaply done at all .  they pack a ton of detail into the images , especially during scenes involving mummies rendered completely by cgi .  the computer special effects makes for some brilliant scenes , such as unnerving moments involving flesh-eating scarab beetles and moments where the mummy goes after the people who freed him ( after all , those who took his artifacts are cursed ) .  unfortunately , the film attempts to be way too much in such a short span of time , becoming a tug of war for control between genres .  on one hand , it is a fast paced action film .  on the other , it's a frightening horror film .  and on the side , it's a hilarious comedy .  ideally , for it to be successful , the film would have to focus on one the more action oriented aspect , with one character serving as the comic relief ( that would be johnson ) .  the problem with this film is that it has at least three characters serving as comic relief , with rick occasionally delivering his witty one-liner .  if they were trying to make a horror-action-comedy , it would have helped if it were established early on in the film , but unfortunately , with the backstory of imhotep's entombment , that would be impossible .  and then there's the slapstick fight scene between a sword-wielding rick and an army of mummies .  while really well done , it had the feel of slapstick comedy .  replace the sword with a chainsaw and you'd effectively have ash fighting zombies in army of darkness .  while entertaining and funny , it feels really out of place .  but at least it's a break from the naive heroes that brendan fraser has been playing a lot of .  on the whole , the movie is pure popcorn fare from beginning to end , entertaining the audience .  but i must end my review with a plea to movie theatre owners . . . turn  the sound down !  while a lot of theatres have good sound systems , my ears were almost ringing as i walked out of the theatre ( the sound is particularly irritating and will make you crap your pants if you're not careful ) .  when they were showing the trailer for the upcoming schwarzenegger film , " end of days " , it was so loud i couldn't make anything out .   & 0 & 0.766667 & 0.700000 & The overall sentiment of the text is slightly positive, with praise for the visuals and CGI, but criticism for the film's attempt to incorporate multiple genres, excessive comedic relief, and loud sound in the theater. & 0.446386 \\
1.000000 & lean , mean , escapist thrillers are a tough product to come by .  most are unnecessarily complicated , and others have no sense of expediency -- the thrill-ride effect gets lost in the cumbersome plot .  perhaps the ultimate escapist thriller was the fugitive , which featured none of the flash-bang effects of today's market but rather a bread-and-butter , textbook example of what a clever script and good direction is all about .  the latest tony scott movie , enemy of the state , doesn't make it to that level .  it's a true nineties product that runs like greased lightning through a maze of cell phones and laptop computers , without looking back .  although director scott has made missteps in the past , such as the lame thriller the fan , he's generated a good deal of energy in pictures like crimson tide and top gun .  that vibrant spirit is present here , shown in well-timed and carefully planned chase scenes that give the movie an aura of sheer speed .  enemy of the state also features an unprecedented use of amazing cinematography -- director of photography daniel mindel throws a staggering amount of different views , angles , lenses , and film stocks at the audience that goes a long way toward involving the audience in the movie .  enemy is truly a visual experience , and that's only one of the reasons it's such a fun watch .  the movie lights up with an aging senator visited by nsa deputy chief thomas reynolds ( jon voight ) .  reynolds wants a new communications act passed to allow the government free reign in the use of surveilance equipment , but the senator plans to bury the bill in committee .  reynolds has the senator offed , but not before the murder is caught on a naturalist's camera .  by an extremist chain of events , the tape ends up in labor lawyer robert dean ( will smith ) 's posession , and it's not long before he's running from reynolds' cronies .  it's only with the help of an ex-spook named brill ( gene hackman ) that dean is able to get to the bottom of things .  the acting is top notch , and the three principles - smith , hackman , and voight - are generally more mature and excellent all around .  smith puts aside the wisecracking act and becomes a normal human being ; voight tones down the amount of sneer he puts into his character for greater ominpotence ; and hackman is simply over the top in the mysterioso role .  smith's regular joe comes off particularly well , as he runs from authorties for reasons that he knows not .  the supports are also in fine form , lending credibility to the main roles and advancing the plot in key areas .  this is , for the holiday crowd , the hot ticket ; as well as anyone looking for a serving of genuine action in a market that is otherwise lacking .   & 0 & 0.833333 & 0.866667 & The text evaluates the movie "Enemy of the State" with both positive and negative statements. While it praises the vibrant spirit, well-timed chase scenes, amazing cinematography, and top-notch acting, it also criticizes the unnecessary complexity and lack of sense of expediency in most escapist thrillers. Although it falls short of being the ultimate escapist thriller, the positive aspects of the film outweigh the negative ones, resulting in a slightly positive sentiment. & 0.964172 \\
\bottomrule
\end{tabular}
