\begin{tabular}{r{1cm} p{0.4in} r{1cm} p{0.4in}}
\toprule
sentiment & sentence & sentiment_score & explanation \\
\midrule
1.000000 & good films are hard to find these days .  great films are beyond rare .  proof of life , russell crowe's one-two punch of a deft kidnap and rescue thriller , is one of those rare gems .  a taut drama laced with strong and subtle acting , an intelligent script , and masterful directing , together it delivers something virtually unheard of in the film industry these days , genuine motivation in a story that rings true .  consider the strange coincidence of russell crowe's character in proof of life making the moves on a distraught wife played by meg ryan's character in the film -- all while the real russell crowe was hitching up with married woman meg ryan in the outside world .  i haven't seen this much chemistry between actors since mcqueen and mcgraw teamed up in peckinpah's masterpiece , the getaway .  but enough with the gossip , let's get to the review .  the film revolves around the kidnapping of peter bowman ( david morse ) , an american engineer working in south america who is kidnapped during a mass ambush of civilians by anti-government soldiers .  upon discovering his identity , the rebel soldiers decide to ransom him for $6 million .  the only problem is that the company peter bowman works for is being auctioned off , and no one will step forward with the money .  with no choice available to her , bowman's wife alice ( ryan ) hires terry thorne ( crowe ) , a highly skilled negotiator and rescue operative , to arrange the return of her husband .  but when things go wrong -- as they always do in these situations -- terry and his team ( which includes the most surprising casting choice of the year : david caruso ) take matters into their own hands .  the film is notable in that it takes this very simple story line and creates a complex and intelligent character-driven vehicle filled with well-written dialogue , shades of motivation , and convincing acting by all the actors .  the script is based on both a book ( the long march to freedom ) and a magazine article pertaining to kidnap/ransom situations , and the story has been sharply pieced together by tony gilroy , screenwriter of the devil's advocate and dolores claiborne .  the biggest surprise for me was not the chemistry between crowe and ryan , but that between crowe and david caruso .  dug out from b-movie hell , caruso pulls off a gutsy performance as crowe's right hand gun while providing most of the film's humor .  ryan cries a lot and smokes too many cigarettes , david morse ends up getting everyone at the guerilla camp to hate him , and crowe provides another memorable acting turn as the stoic , gunslinger character of terry thorne .  the most memorable pieces of the film lie in its action scenes .  the bulk of those scenes , which bookend the movie , work extremely well as establishment and closure devices for all of the story's characters .  the scenes are skillfully crafted and executed with amazing accuracy and poise .  director taylor hackford mixes both his old-school style of filmmaking with the dizziness of a lars von trier film .  proof of life is a thinking man's action movie .  it is a film about the choices men and women make in the face of love and war , and the sacrifices one makes for those choices -- the sacrifices that help you sleep at night .   & 0 & The overall sentiment is slightly negative due to initial comments about the scarcity of good films, but is mostly positive due to the review of "Proof of Life" as a rare gem with strong acting, intelligent script, masterful directing, well-written dialogue, and skillfully crafted action scenes. \\
1.000000 & almost a full decade before steven spielberg's saving private ryan asked whether a film could be both " anti war " and " pro-soldier " , john irvin's hamburger hill proved it could .  lost in the inundation of critical acclaim that greeted oliver stone's platoon , this excellent film was dismissed as " too militaristic " .  it's hard to understand exactly why---unless irvin , in assembling his motley collection of young men who for predictable ( and often naive ) reasons " chose to show up " for the vietnam debacle , ---has refused to present us with the stone killer , drug-stoked psycho and ruthless opportunist who have become to vietnam war epics what " the polack , the hillbilly and the kid from brooklyn " became to wwii movies .  hamburger hill , based on a true story , is not an easy film to watch .  there is a scene that will have graying anti-war activists squirming in their seats , or moved to genuine tears .  and the climactic final assault on the " hill " in question is visually confusing .  gristly realities are presented in brief flashes , as if the brain dared not acknowledged what it had encountered .  and in the mud and smoke officer and enlistee , veteran and " newbie " , black soldier and white , become almost indistinguishable from each other , as they do in the chaos of actual combat .  the acting throughout is solid with an absolutely stellar performance rendered by courtney b . vance as doc--in a role that will have many flatly disbelieving that this is same actor they cheered as " seaman jones " in mctiernan's red october .  if you've seen private ryan , you owe it to yourself to see hamburger hill--if only to determine that the all the valour and horror of spielberg's vision was as present in the ashau valley as it was at omaha beach .   & 0 & The overall sentiment of the explanations about "Hamburger Hill" is mixed, with some reviewers expressing a negative sentiment due to the film's depiction of war and a confusing final assault scene, while others praise its anti-war and pro-soldier message, realistic portrayal, and strong performances. \\
1.000000 & did you ever wonder if dennis rodman was actually from this planet ?  or if sylvester stallone was some kind of weird extra-terrestrial ?  i used to think that about my 7th grade english teacher , ms . carey .  but after seeing this movie , they may have confirmed my suspicions .  as the story goes , at any time , there are over a thousand aliens living among us here on earth .  the men in black ( mib ) are the watchdogs that oversee the cosmic citizens , guardians of our beloved planet from nasty-tempered aliens , and secret service to the stars .  based in new york city ( where weird is the norm ) , the mib organization gives human form to our space-faring emigrants so that they may walk and live among us unnoticed .  but to enforce the laws of earth , the mib carry weapons that are powerful enough to meet or exceed destruction quotas in one single blast .  they carry other-worldly technology to erase people's short-term memory when common folk see the mib in action .  and their best leads on cosmic things-gone-awry are the supermarket tabloids .  little do we know that there are much stronger battles of good v . evil going on in the depths of space .  one of the aliens-as-human on this planet is an important diplomat that is carrying something very precious .  it holds the 'key' , literally , to universal peace .  a giant cockroach-like alien soon arrives on the planet and steals this 'key' .  in the wrong alien hands ( flippers ?  mandibles ?  tentacles ? ) , it can be used as a weapon .  therefore , it must be recovered and returned to it's rightful owners .  otherwise , to ensure universal safety , earth will be destroyed , along with the 'key' .  now , it's the mib who must prevent this catastrophe .  the mib agents on the case are " k " , played by tommy lee jones .  he is crustier than burnt toast and even more serious than al gore .  the stars in the sky no longer spark wonder in his eyes .  he is accompanied by a flippant rookie , " j " , played by will smith .  but , despite this shoot-em-up , protect-earth-from-destruction premise , this is nothing at all like a typical summer action movie .  and , this isn't an independence day knockoff .  rather , this is a stylishly offbeat sci-fi comedy that pokes fun at what the government always denies ? that there are real aliens that live here , and that the government does its darndest to cover them up .  but to give it some sense of excitement and to keep it within the parameters of the summer movie recipe , there must be some kind of earth-hangs-in-the-balance scenario .  yet , this movie is very appealing .  the abundance of wierdness ( talking aliens , pee-wee atomizers , a mortician who 'lives' for her work , and lots of yucky bugs and slime-splattering galore ) , is played straight , like as if this were normal ( of course , we are in nyc ) .  it gives it a deadpan feel , which makes it all the more funnier and odder .  jones plays the venerable seen-it-all agent with seriousness and maturity .  smith is likeable and makes a great comic partner to jones' straight man routine .  they click like dorothy's ruby red shoes .  the look and feel of the movie is made even better with direction from barry sonnenfeld ( the addam's family ) .  this guy has a knack for 'gothic' comedy , and successfully transfers his macabre sense of humor onto the screen .  and , an appropriate dose of special effects helps to bolster the oddness of their task without diverting attention from the human actors .  the story moves well , and before you know it , the end credits are already rolling !  the result is 100 minutes worth of fun in the form of ewwwws and blechhhs , aaaahhhs and wows .  let the men in black protect and color your world .   & 0 & The text describes a light-hearted and enjoyable sci-fi comedy movie that pokes fun at the government's denial of aliens and praises the performances of the actors, while also mentioning suspicions about strange creatures on Earth and the potential destruction of the planet. \\
1.000000 & trailing the success of brit humour in the movie industry with the likes of the semi-dramatic the commitments and the nearly slapstick a fish called wanda , the full monty is one film which delivered the depth of the former and the humour magnitude of the latter .  the film opens with a narrated documentary reel showing the improving economic and living standards of sheffield in the 60's , then cut to the harsh reality of the present .  sheffield has become some sort of a semi-slum with the only visible increase in anything was the amount of layoffs from steel-factories ; a once flourishing industry .  gaz ( carlyle ) spends most of his time in the worker's club , a sort of place where jobless people sit around to wait for job offers .  he and his `plump' friend dave ( addy ) as well as their former foreman gerald ( wilkinson ) have been sitting around the club for months without any `call for duty' .  what seemingly was just a passing hard time for gaz suddenly transformed itself into desperation when he cannot afford the 700 pounds of child-support money to his ex-wife .  suddenly finding himself facing the possiblity of losing custody of his son , he goes on to concoct an enterprising wild-idea to get the money he desperately requires .  dave and gerald too have problems of their own .  on top of his lack of employment , dave is also faced with the paranoia of his wife leaving him because of his current financial state as well as his`plump' appearance while gerald have been cheating on his wife by not telling her of his layoff for as long as 6 months , leaving home for `work' when he was actually a long-standing member of the worker's club .  there we have it ; people with real problems and a not-so-practical solution for them .  gaz manages to get the support from them and sure enough they managed to get a few other poor jobless blokes to join in their gag : to perform a strip-act at the local pub .  in gaz's own words while looking at an add for a strip-act featuring hunky `beautiful' men , `if women are willing to pay for this , they sure as well will pay for real-men' .  the full monty is very simple in nature and not a trace ambitious at all .  it is a small film which delves in the harsh-reality of unemployment and the desperation it drives people to .  it also is a very light movie to watch despite the theme it delves in because it has a lot of human-factor going for it .  never for once , was it trying to be manipulative at all , the full monty was true to itself all the way .  while the hilarity level of this film soared unexpectedly high , audience will find that they are in no way being cheated of a laugh , something lacking in many comedy films ( wannabes ? ) from hollywood .  other than carlyle , who acted as the psychotic begbie in trainspotting , the rest of the actors were definitely very new to me on-screen .  however , the acting presented in this film is more than satisfying .  carlyle's work here is a stark contrast to his character in trainspotting , implying real acting skill and flexibility on-screen .  kudos to the people involved in this film , especially to director peter cattaneo for being able to put together an excellent film which is has so much potential in ending up as your just-another-striptease .  already a critical and financial success for such a small film , the full monty is a humble film which deserves applause throughout .  while titanic managed to feast the eyes and soul in an epic way , the full monty gives one the same satisfaction ? ? . in  half the time .   & 0 & The text contains positive sentiments towards the film "The Full Monty," praising its success in British humor, comedy and human-factor, as well as its portrayal of real problems faced by unemployed individuals in a humorous and light-hearted way, commending the acting and direction. \\
1.000000 & recently i read 4 reviews of pleasantville-one from entertainment weekly , one from a newsgroup , and two from different online resources .  each review compared this film to the truman show .  why ?  the only reason people compared pleasantville to truman is due to the fact that their lives are on television .  other than that , the two movies are completely different from each other .  reviewers seem to love to pick one movie ( i . e .  the truman show ) , obsess over it , and make it a guideline for other movies .  when pleasantville and edtv opened , all critics seemed to treat the two films like they were subjects of king truman .  the same goes for the thin red line , as well .  when it opened in decemeber , probably 98% of reviews compared it to stephen spielberg's saving private ryan , giving thin line no chance whatsoever .  critics loved private ryan so much , that they automatically decided no movie is good enough to reach its standards .  this is why i think pleasantville was underrated .  if the truman show had never been made , pleasantville probably would have received better reviews , done better at the box office , and would be remembered after it's long been on the video store shelves .  sure , it wasn't perfect , and it wasn't very believeable , but neither was the wizard of oz or star wars , and there stands two of the most prominent movies in history .  with the exception of don knotts as the annoying " tv repairman " the film is cast perfectly : the ice storm's tobey maguire is david in real life ; he watches the old 50's sitcom " pleasantville " to escape from his feuding parents .  his twin sister , jen ( reese witherspoon ) is a popular slut who ( steriotypically ) smokes and wears revealing clothes .  but what makes her character believeable is the dialogue-those excessive 90's terms such as " like " , " whatever " , and " cool " make her sound like the total dimwit she's supposed to be .  the twins , with a little help from a tv repairman ( casting don knotts in this role was obviously a cameo-esque cast rather than a true acting one ) , are sucked into pleasantville by a magical new remote .  their new parents , played by william h . macy and joan allen , are perfect in every way .  their new names are bud and mary sue , and right away they start to corrupt the town of pleasantville .  after the siblings fit comfortably in their new roles , the movie begins to take form .  it's just like watching a sitcom in itself ; you don't want it to end .   " mary sue " has sex with her date , the school basketball champion ; when he tells all his friends about it , they stop preforming perfectly at basketball , and things start to take color .  soon we learn that these people don't know how to express true emotions , and when they learn to do so , they eventually turn into color .  this eventually sparks a racial war between the " coloreds " and the " black & whites . "  entertainment weekly reviewer lisa schwarzbaum claims the movie has , " none of the depth , poignance , and brilliance of the truman show . . . "  yes , the truman show was maybe a little more intelligent .  but the racial setting-intensified by the forcing of the coloreds to sit in the upper box of the courtroom-certainly classifies as deep , considering it's what i least expected .  gary ross did not try to create an intelligent , award masterpiece-he just tries to convey moral messages within his work .  take for example big and dave , his past films .  big-be careful what you wish for .  dave-good or bad lying is still lying .  did these movies win big awards ?  no , but they won the heart of millions .  pleasantville could have done that too , if it wasn't for snotty reviewers who set precedents with preceeding films .  pleasantville : a- ad2am " i almost lost my nose . . . and  i like it .  i like breathing through it . "  -jack nicholson , chinatown   & 0 & The overall sentiment of the explanations is slightly negative, with frustration expressed towards critics comparing "Pleasantville" to other films, such as "The Truman Show". The author believes that "Pleasantville" was underrated and could have achieved better reviews without the constant comparisons. The cast's performances are praised, but the film's flaws are also acknowledged. \\
1.000000 & note : some may consider portions of the following text to be spoilers .  be forewarned .  during the three years since the release of the groundbreaking success pulp fiction , the cinematic output from its creator , quentin tarantino , has been surprisingly low .  oh , he's been busy -- doing the talk show circuit , taking small roles in various films , overseeing the production of his screenplay from dusk till dawn , making cameo appearances on television shows , providing a vignette for the ill-fated anthology four rooms -- everything , it seems , except direct another feature-length film .  it's been the long intermission between projects as well as the dizzying peak which pulp fiction reached which has made mr . tarantino's new feature film , jackie brown , one of the most anticipated films of the year , and his third feature film cements his reputation as the single most important new american filmmaker to emerge from the 1990s .  things aren't going well for jackie brown ( pam grier ) .  she's 44 years old , stuck at a dead-end job ( " $16 , 000 a year , plus retirement benefits that aren't worth a damn " ) as a flight attendant for the worst airline in north america -- and she's just been caught at the airport by atf agent ray nicolette ( portrayed with terrific childlike enthusiasm by michael keaton ) and police officer mark dargus ( michael bowen ) smuggling $50 000 from mexico for gun-runner ordell robbie ( samuel l . jackson ) , who has her bailed out by unassuming bail bondsman max cherry ( robert forster ) .  the loquacious ordell , based out of a hermosa beach house where his horny , bong-hitting surfer girl melanie ( bridget fonda ) and agreeable crony louis gara ( robert de niro ) hang out , operates under the policy that the best rat is a dead rat , and he's soon out to silence jackie brown .  meanwhile , the authorities' target is ordell , and they want jackie to help them by arranging a sting to the tune of a half-million dollars .  only through a series of clever twists , turns , and double-crosses will jackie be able to gain the upper hand on both of her nemeses .  although jackie brown marks mr . tarantino's first produced screenplay adaptation ( based on the elmore leonard novel " rum punch " ) , there's no mistaking his distinctive fingerprints all over this film .  while he's adhered closely to the source material in a narrative sense , the setting has been relocated to los angeles and the lead character's now black .  in terms of ambiance , the film harkens back to the 1970s , from the wall-to-wall funk and soul music drowning the soundtrack to the nondescript look of the sets -- even the opening title credit sequence has the echo of vintage 1970s productions .  the opening sequence featuring ms . grier wordlessly striding through the lax , funky music blaring away on the speakers , is emblematic of films of that era .  the timeframe for the film is in fact 1995 , but the atmosphere of jackie brown is decidedly retro .  of course , nothing in the film screams 1970s more than the casting of pam grier and robert forster as the two leads , and although the caper intrigue is fun to watch as the plot twists , backstabbing , and deceptions deliciously unfold , the strength of jackie brown is the quiet , understated relationship developed between jackie and max ; when they kiss , it's perhaps the most tender scene of the year .  tenderness ? in a quentin tarantino film ?  sure , there've been moments of sweetness in his prior films -- the affectionate exchanges between the bruce willis and maria de madeiros characters in pulp fiction and the unflagging dedication shared by the characters of tim roth and amanda plummer , or even in reservoir dogs , where a deep , unspoken bond develops between the harvey keitel and tim roth characters -- but for the most part , mr . tarantino's films are typified by manic energy , unexpected outbursts of violence , and clever , often wordy , banter .  these staples of his work are all present in jackie brown , but what's new here is a different facet of his storytelling -- a willingness to imbue the film with a poignant emotional undercurrent , and a patience to draw out several scenes with great deliberation .  this effective demonstration of range prohibits the pigeonholing of mr . tarantino as simply a helmer of slick , hip crime dramas with fast-talking lowlifes , and heralds him as a bonafide multifaceted talent ; he's the real deal .  this new aspect of mr . tarantino's storytelling is probably best embodied in a single character -- that of the world-weary , sensitive , and exceedingly-professional max cherry , whose unspoken attraction to jackie is touching .  mr . forster's nuanced , understated performance is the best in the film ; he creates an amiable character of such poignancy that when he gazes at jackie , we smile along with him .  much press has been given about the casting of blaxploitation-era icon pam grier in the lead , with the wags buzzing that mr . tarantino may do for her what his pulp fiction did to bolster john travolta's then-sagging career .  as it turns out , ms . grier is solid in the film's title role , although nothing here forces her to test her range .  i do have to take exception to the claim that this film marks her career resurrection , though -- she's been working steadily over the years , often in direct-to-video action flicks , but also in such recent theatrical releases as tim burton's mars attacks !  and larry cohen's original gangstas ( where she first teamed up with mr . forster . )  of course , it's true that her role here was a godsend -- a meaty a part as this is rarity for * any * actress , let alone one of her age and current status in the industry .  while jackie brown may disappoint those looking for another pulp fiction clone , it marks tremendous growth of mr . tarantino as a director whose horizons are rapidly expanding , and whose characterizations have never been better .  and while the film's narrative doesn't really warrant a running time of 155 minutes , it's filled with such sumptuous riches , ranging from the brashness of the vivid soundtrack to entertaining , inconsequential conversations between the characters , that there wasn't an unengaging moment .  with an impressive trio of feature films under his belt , it'll be interesting to see what he tries next .   & 0 & The overall sentiment of the explanations is slightly negative, with some positive elements mentioned such as the growth and improved characterizations in Quentin Tarantino's new film, "Jackie Brown," but also criticisms of the long intermission between projects, disappointment of those expecting another Pulp Fiction, and the narrative not warranting the running time of the film. \\
1.000000 & with many big-budget science fiction films , great ideas are often wasted by bad scripts , cheesy plot twists , and , terrible acting .  the fifth element , the abyss , and godzilla had great concepts squandered by bad acting , writing , or both .  at first glance , the matrix , larry & andy wachowski's sci-fi/kung-fu/shoot-em-up spectacular , looks like a prime candidate to join the list of high-concept bad movies , especially with dopey keanu reeves in the leading role .  but despite high levels of cheese and gaudy dialogue , the matrix works .  it's an uncanny blend of action and surreal fantasy that borrows from dozens of other films ( most obviously the terminator films , star wars , and total recall ) , but remains refreshingly original and interesting throughout .  reeves plays neo , a computer hacker who stumbles into an initially bewildering set of encounters with " trinity " ( carrie-anne moss ) , a rival hacker who has supernatural powers and stunning good looks , a unseen , omniscient cult figure known as " morpheus " ( lawrence fishburne ) , and a trio of creepy men in dark suits who act like irs agents from hell .  neo quickly learns that the world he has known all his life is not what it seems .  moreover , he discovers that the grim , bleak world he is introduced to has been waiting for him to save it .  neo is reluctant to assume the role of messiah , with grave doubts that he is actually " the one " prophesized to come and save the world .  reluctant or not , what follows is a mix of hong kong-style slow-motion shootouts , surreal dream sequences , high speed chases , and comic book kung-fu fights .  the film also raises interesting philosophical questions about reality .  how exactly do you know what is real and what is in your mind ?  and if the real world were much harsher and grim than a fantasy one you were living in , would you want to face the true world or continue to live in a more comfortable illusion ?  unfortunately , the matrix doesn't provide many answers to the questions it raises , but at least it puts some ideas behind all the explosions , shoot-outs , and flying roundhouse kicks .  perhaps stealing the entire show in the matrix is australian actor hugo weaving , who plays agent smith , the leader of the creepy agents in sunglasses and suits who seem capable of being anywhere and doing anything to stop neo and friends from destabilizing the matrix .  weaving's stony appearance , deadpan voice , and chilling comments put a grim human face on the haunting , evil technological force that controls the matrix .  like robert patrick as the shapeshifting t-1000 terminator in terminator 2 : judgement day , weaving is more frightening than any monstrous alien or homicidal robot because , despite his power and seeming invincibility , he looks ordinary , even scrawny .  weaving embodies his role with a memorably chilling presence .  the film features some truly breathtaking special effects and stylish cinematography .  it seems to be deliberately kitchy , with stylized fight sequences directed by hong kong stunt specialist wo ping that reportedly required months of martial arts training by the actors .  in a few scenes , the posing and posturing is unintentionally funny .  just seeing keanu reeves engaging in serious kung fu is a bit jarring .  however , with the comic-book style and tone of the film , you can sustain willing disbelief long enough to enjoy the ride without losing patience .  the matrix isn't a classic .  its open-ended and confusing conclusion raises more questions than it answers .  it fails to resolve many of its own plot twists and philosophical questions .  it also relies on some conventional sentimentality to save the hero - the kind of shmaltzy feel-good goo that most of the film avoids .  reportedly , the producers have high hopes for the film being the first in a trilogy and have already begun work on the story for the sequel .  unfortunately , audiences deserve a little more than a vague sense that the story will continue in the future .  the matrix is an fun , enjoyable diversion , like a big puff of cotton candy at a carnival .  just don't be surprised half an hour later when you still feel a little hungry .   & 0 & The sentiment of the text is slightly negative, as the author criticizes big-budget science fiction films for their flaws but acknowledges that "The Matrix" stands out as an original and enjoyable film. \\
1.000000 & note : some may consider portions of the following text to be spoilers .  be forewarned .  milos forman's first film since the ill-fated valmont , columbia's the people vs .  larry flynt , is a vastly entertaining ( if not particularly enlightening ) biopic of hustler publisher and self-made millionaire larry flynt , who became an unlikely champion of freedom of speech rights in the united states in the late 1970s and early 1980s .  the film unweaves its tale in a chronological order : we open with young and dirt-poor larry flynt and his brother jimmy , peddling jars of water in true entrepreneurial spirit out in the rural outback of kentucky .  cut to forward in time , where the two flynt brothers , now young men , are running the struggling hustler go-go clubs in cincinnati .  the strip clubs are in a dire financial state , and in a last-ditch effort to salvage the operations , flynt decides to go to a print shop and churn out a promotional newsletter .  this evolved into the adult periodical _hustler_ magazine , creating larry flynt a vast financial empire , and the rest is history .  what sets flynt apart from other publishers is his struggles against those who would have him cease publication of his adult material , and who railed and preached against him - flynt spent time in incarceration and was paralyzed by an assassination attempt - and his driven , single-minded insistence to buck the system and fight for his freedom of expression , regardless of personal cost .  the people vs .  larry flynt also weaves in the bittersweet story of flynt's true love , althea leasure , whom he meets as a dancer in his club and later marries , and who devotedly stands alongside him throughout his trials and tribulations .  considering the serious nature of the film's theme - the importance of the united states' first amendment - the people vs .  larry flynt is surprisingly and wonderfully light-hearted and humourous .  much of the comedy is elicited from larry flynt's outlandish stunts at his courtroom appearances - some of his chosen apparel is hilarious - and for the most part these elements of the film work far better than some of the more dramatic points , such as an uninspiring flynt monologue set at a free speech rally in front of an enormous american flag dealing with the subjectivity of obscenity .  the film's focus is on the flynt's many battles over first amendment rights and freedom of speech , but the heart of the people vs .  larry flynt is the touching love story between flynt and althea .  larry flynt is shown as being occassionally gruff , harsh , and overtly aggressive with his friends and colleagues , but with althea , we see his loving , affectionate side .  there's a scene where flynt tenderly takes his ill wife on a ride on his wheelchair that is heartbreaking .  ultimately , the emotional power that the film hits at its conclusion comes not from his achievements from his battles against censors , but from the strength of flynt and althea's love for each other .  woody harrelson is entirely engaging in what must be certainly a career-topping performance as the irrepressible larry flynt .  harrelson plays flynt with the right mixture of outrageousness and confident stubborness to make him endearing and entirely sympathetic to the audience , and a very compelling protagonist for the film .  courtney love plays althea leasure in a startling turn , completely raw and impulsive .  it's a very solid performance , brash and naturalistic , and love is extremely compelling ; it's difficult to take your eyes off her onscreen , and her chemistry with harrelson is dead-on .  edward norton , as flynt's straight , level-headed lawyer is often upstaged by his flashier co-stars in the people vs .  larry flynt , much as his counterpart lawyer alan isaacman was upstaged by flynt during many of the courtroom scenes , but norton shines in his big scene where he addresses the supreme court in the climactic scene of the film .  one can sense the frustration that norton's character feels when harrelson's free-talking flynt sabotages trial after trial on him by openly speaking his mind , and this results in a heightened emotional punch when norton's isaacman has the opportunity to sway the supreme court judges .  milos forman keeps the film moving - although it runs over two hours , it never drags - and his direction of the film is very effective , eliciting a great deal of empathy for a subject which could be construed by some as extremely sordid and unsympathetic .  there's also a great visual technique which forman uses to indicate the passing of time in one shot , which is both clever and extremely entertaining .  two minor quibbles with the film - it certainly seems like the people vs .  larry flynt is in a rush to get to its main theme , with flynt battling against authorities and the system for his freedom of speech .  consequently , the first thirty minutes of the film , introducing and setting up the characters , seem unduly rushed ; perhaps it is merely due to the fact that these characters are so interesting , but i felt it would have worked better if this route was taken in a more leisurely fashion .  it also felt like there was a distinctive lack of insight into the inner workings of these characters - the film clearly shows what flynt , althea , isaacman , and rev . jerry faldwell did , and on a superficial level some of their motivations , but it never seemed like one could really understand the characters on a deeper level .  for example , why larry flynt was compelled by ruth carter stapleton ( president carter's sister ) to be born-again is a mystery to me .  then again , perhaps it was to him as well .  these two points don't detract greatly from the film .  the people vs .  larry flynt is certainly among the very best studio-released films of 1996 , and works both as a ringing political statement about the importance of freedom of speech and the depths to which larry flynt would go to advance the cause of free expression , and as a touching love story .   & 0 & The sentiment of the review is generally positive, with praise for the entertaining and light-hearted nature of the film and the performances of Woody Harrelson and Courtney Love. However, there are some minor criticisms regarding the rushed introduction of characters and a lack of deeper character insight. \\
1.000000 & all great things come to an end , and the dot-com era embodies that perfectly .  beneath a mound of bankruptcy paperwork lies the remains of a former dot-com darling , the company kozmo . com ,  an online convenience store stocked with ice cream , porn videos , and other basic necessities of a urban dweller , all hand-delivered by couriers within an hour .  designed in 1997 by two college roommates -- joseph parks , a 27 year old goldman sachs banker , and yong kang - kozmo flamed out in three short years , raising more than $280 million in venture capital funding and from partnerships with such bigwigs as starbucks and amazon . com .  by december 1999 , the company boasted 4 , 000 employees in 11 cities , its barking ceo park attracting all kinds of media attention .  the company was set for an ipo in may 2000 . . .  until april 14 , 2000 , the day the stock market took its first big dive , ending the internet era .  by april 13 , 2001 , kozmo was out of money and ceased operations .  unlike the earlier , similar documentary startup . com , which chronicled the rise and fall of another dot-com , govworks , e-dreams focuses both on its original founders , especially park , and on the common folks that ran the day-to-day operations .  the contrast is amazing , showing how a cult persona can convince anyone that any idea is the next big thing .  the film's director , wonsuk chin ( too tired to die ) , expertly juxtaposes upper management company meetings with on-the-spot interviews with the bike messengers , general managers , and floor staff that kept kozmo humming .  the film's images give a backbone to the company and provide an emotional edge to its ultimate demise .  the most satisfying part of the film comes in understanding , to a degree , the expectations of numerous ceos commanding these titanic-type businesses .  in the film's final interviews with park , we learn what happens when the money dries up and backers don't return phone calls .  in the end , the name of the game was profit , and if you couldn't make money , even the dreamers got the axe .  screened at the 24th annual mill valley film festival .   & 0 & Overall, the text conveys a slightly negative sentiment by discussing the downfall of the company Kozmo and the dot-com era, while also incorporating positive elements like the contrast between upper management and employees. \\
1.000000 & kevin smith is like a big kid .  his humor is that of a sophisticated juvenile's .  he grew up idolizing star wars and loves comic books , having also written a few .  he also has a cult following , mostly composed of teenagers , college students , and smith's own fellow adolescent-minded grownups .  smith is hilarious in person and in writing , but when he tries to be earnest and moralize , that is when he goes wrong .  kevin smith is a better writer than director , and he'll be the first to tell you that .  that might also be part of the reason why his moralizing comes across as so heavy-handed .  great directors show us their theses instead of having the characters sermonize them .  this was true in the overrated chasing amy , and it is true for dogma as well .  that is not to say smith's message is a bad one .  in dogma , smith tells us that problems arise when people believe beyond any doubt that their insight into god and god's desires is superior to anyone else's .  basically , dogmatism is bad .  changing the minds of the dogmatic is virtually impossible , and since the dogmatic believe that they have special insight , they also know what is best for you , whether you like it or not .  this is not exactly a new message in movies ( see inherit the wind ) , but i have no problems with recycling old ones , particularly since dogma's protesters are proving smith's point .  smith's own problem with delivering this message is that he beats us over the head with it like we are reading a dogma for dummies book .  but this is smith's personality , and his simplistic views neglect such adult issues as how does one interpret the bible ( or koran , etc . ) correctly ( or if there even is a correctly ) and how one settles disputes of heretofore dogmatic concerns .  the story concerns abortion clinic worker bethany ( linda fiorentino ) being chosen by voice of god , metatron ( alan rickman ) , to prevent the destruction of the universe by two fallen angels , bartleby ( ben affleck ) and angel of death , loki ( matt damon ) .  along the way , forgotten thirteenth apostle , rufus ( chris rock ) , stripper muse serendipity ( salma hayek ) , and slacker duo jay and silent bob ( jason mewes and kevin smith himself ) come to bethany's aid .  fallen muse , azrael ( jason lee ) , proves to be the behind-the-scenes manipulator for all the chicanery .  the logical but convoluted plot only exists as an excuse for the jokes and to make smith's points , and in itself , has little dramatic momentum .  among the supposedly outrageous claims made by the film is that god is a woman , jesus was black , and the bible was written by a bunch of racist , misogynistic white men .  of course , kevin smith does not necessarily subscribe to these ideas himself .  they are a metaphor for the fears and insecurities of the dogmatic .  smith says as much in his amusing disclaimer that precedes the movie .  when harvey weinstein asked smith to put it into the film before cannes , smith thought it might give validation to protesters' claims that the film was sacrilegious , but then he rethought it and turned the disclaimer into a joke .  the film's humor is uneven .  some parts are very funny as when bethany goes for a fire extinguisher when metatron makes a burning-bush kind of entrance .  but many of the film's jokes just bomb , as in virtually anything involving salma hayek's serendipity .  also , some of the jokes can be seen coming from a mile away .  still , smith keeps the zingers coming at a sufficiently rapid pace .  among the actors , fiorentino and rickman stand out by far .  fiorentino virtually by herself gives the film emotional weight .  damon and affleck are fairly lackluster .  rock and hayek exist in the film pretty much only as comic relief as are mews and smith .  but the latter duo fare much better because jay and silent bob , who recur in all of smith's movies , are much more in line with smith's brand of humor .  bud cort , george carlin , janeane garofalo , guinevere turner , and alanis morissette all make cameo appearances .   & 0 & The sentiment of the text towards Kevin Smith and his work is mixed, leaning slightly towards a balanced or neutral sentiment with both positive aspects such as his humor and cult following, as well as negative aspects such as heavy-handed moralizing and directing flaws. \\
\bottomrule
\end{tabular}
