\begin{tabular}{r{1cm} p{0.4in} r{1cm} p{0.4in}}
\toprule
sentiment & sentence & sentiment_score & explanation \\
\midrule
0 & A cheap and cheerless heist movie with poor characterisation, lots of underbite style stoic emoting (think Chow Yun Fat in A Better Tomorrow) and some cheesy clichés thrown into an abandoned factory ready for a few poorly executed flying judo rolls a la John Woo. & 0 & The explanations indicate a clear negative sentiment towards the movie, highlighting the use of descriptive words such as "cheap," "cheerless," and "poor," as well as criticisms of characterisation and poorly executed action sequences. The mention of John Woo and Chow Yun Fat in a Better Tomorrow provides a slight hint of positivity, but overall the sentiment remains negative. \\
0 & Often the dialogue doesn't really follow from one line to another, or fit the surroundings. & 0 & The explanations express a negative sentiment towards the text due to a lack of coherence in the dialogue and a disconnect from the surroundings, suggesting a flaw in the writing. \\
0 & My only problem is I thought the actor playing the villain was a low rent Michael Ironside. & 0 & The user expresses a mixed sentiment towards the actor playing the villain, comparing them unfavorably to Michael Ironside, but the overall tone is slightly positive with a hint of disappointment. \\
0 & It didn't make me scared, horrified, or make me sympathetic towards the characters; it was simply annoying. & 0 & The explanations indicate that the text expresses a negative sentiment and that the user found it annoying, lacking enjoyment or connection with the content. \\
0 & It was just too horrible. & 0 & The explanations suggest that the text expresses a highly negative sentiment with intense emphasis on the negativity and distress. \\
0 & The poor actors. & 0 & The explanations point towards a slightly negative sentiment due to the use of the phrase "the poor actors," which implies difficulties or lack of performance, although there is a slight positive sentiment suggested by the word "poor" being interpreted as sympathy or empathy. \\
0 & It failed to convey the broad sweep of landscapes that were a great part of the original. & 0 & The sentiment of the prompt is negative (-0.6) due to the disappointment expressed in the failure to convey landscapes effectively in the original work. The explanation provides a concise and accurate analysis of the prompt with a confidence rating of 0.8, demonstrating understanding of the user's sentiment and acknowledging their dissatisfaction. \\
0 & It will drive you barking mad! & 0 & The phrase "drive you barking mad" conveys a strong negative sentiment, indicating extreme frustration or irritation. \\
0 & The casting is also horrible, cause all you see is a really really BAD Actors, period. & 0 & The consensus from the explanations is that the casting of the movie is terrible, with actors being described as bad, which contributes to a strong negative sentiment towards the movie. \\
0 & Of course, the acting is blah. & 0 & The explanation consistently indicates a negative sentiment towards the acting, with the use of the word "blah" suggesting a lackluster performance. \\
\bottomrule
\end{tabular}
