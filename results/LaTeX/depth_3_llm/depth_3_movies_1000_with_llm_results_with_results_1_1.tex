\begin{tabular}{r{1cm} p{0.4in} r{1cm} p{0.4in}}
\toprule
sentiment & sentence & sentiment_score & explanation \\
\midrule
1.000000 & as the film opens up , expectant unwed mother sally ( played by drew barrymore ) encounters her baby's father in a fast food drive-through window .  after he gets his milkshake , he drives off , only to be pursued by a military helicopter .  from this moment forward , you know that this isn't going to be your run-of-the-mill romantic comedy .  home fries tells the tale of the relationship between sally and dorian ( played by luke wilson ) , a helicopter pilot who has a different kind of connection with sally . . . more  than he would care to admit ( hint : it has something to do with the father of sally's unborn child ) .  after taking a personal ( and professional ) interest in her , he finds work at the local fast food restaurant at which she works , and the fun goes on from there .  this movie is a lot of fun .  it is comedic on two levels , as a romantic comedy and a dark comedy .  the romantic moments are sweet , yet not sappy .  one of the more poignant moments is when dorian accompanies sally to lamaze classes .  however , the dark comedic moments are more memorable .  this is also a tale of revenge , and a funny one at that .  the father of sally's unborn child also happened to be a married man , whose wife is played by catherine o'hara .  she does a wonderful portrayal of a jealous wife who is targeting the person who had an affair , and is not above getting other people to do her dirty work .   ( my apologies if this sounds cryptic , but there is too much of the plot to give away ) .  rounding out the cast are jake busey as angus , dorian's overzealous brother and daryl mitchell as roy , dorian's trainer at the fast food restaurant .  the dialogue is sharp and not filled with sappy overtones , even with the romantic nature of the plot .  if there are any complaints , it is over drew barrymore's performance .  while she does light up the screen , there are moments when she doesn't seem all that convincing .  the movie was filmed in texas , and most of the characters have southern texan accents .  drew barrymore's accent tends to come and go ( luke wilson doesn't seem to have this problem ) .  as well , she seems awfully agile for a woman that is eight months pregnant .  still , this movie is one that will please many viewers .  with plot twists galore and hilarious dark moments , this is sure to entertain .   & 1 & The movie "Home Fries" is a fun and entertaining film with both comedic and dark moments, praised for its sharp dialogue and memorable dark comedic moments. There are minor criticisms of Drew Barrymore's performance and inconsistencies with her accent, but overall the sentiment is positive. \\
1.000000 & i can already feel the hate letters pouring in on this one , folks .  i loved the wedding singer .  but it gets worse .  if i so much as mention titanic in the same sentence .   .   .  hell , same paragraph as the wedding singer , i'm bound to be lynched .  well lynch me then , because i thought adam sandler and drew barrymore had the most convincing chemistry in recent memory , with titanic as the only exception .  right away , you're ready to discount my review , aren't you ?  you're thinking that i can't possibly know what i'm talking about .  after all , this is adam sandler right ? ? ?  well , there's more .   .   .  i actually got choked up , and more than once .  yes , the man who brought us billy madison and happy gilmore was able to bring sincere tears to my eyes .  but before you shrug me off as an imbecile , i urge you to see the movie and see for yourself .  the wedding singer is the story of robby hart ( sandler ) , a 1985 rock star wannabe whose soul achievement in life has been performing cover tunes at wedding receptions for the past five years or so .  but when his own wedding results in a no-show for the bride ( angela featherstone as linda ) , robby begins to re-examine his life , and wonder why true love doesn't exist for people like him .  in the process of performing at these receptions , robby meets a waitress named julia , played with wholesome sweetness by drew barrymore , whose reputation is anything but this character .  julia too is engaged , and when robby finds himself without a wife , his free time allows him to help her plan her own wedding , seeing as how her fiance , the " miami vice " loving glenn ( matthew glave ) , doesn't seem all that interested .  as the time goes by , we learn the reason glenn is do dispassionate about the wedding - he's merely marrying julia to prevent a breakup , something he wants to avoid even if he does sleep around with tons of other women .  it's obvious that robby is going to fall for julia and feel a strong need to rescue her , but what is a lighthearted romantic comedy for ?  this isn't sleepless in seattle , but i'd actually compare it to that before i'd compare it to happy gilmore .  while the wedding singer maintains some of that post-snl cornball humor , its sincerity and charm carry it a very , very long way .  sandler is great in his role , and yes , if you allow it , you can take him seriously .  barrymore seems perfect along side him , and although the two more likely represent gen-x moronism and party-girl ditzism ( respectively ) , the two go together like bread and butter as a mid-eighties couple .  one scene in particular , where julia ponders the future in a mirror , is so well done , i was on the verge of tears .  unfortunately , sandler is going to make a lot of people shrug this one off as a no-brainer , but it's got so much surprise substance packed inside that i'm encouraging everyone to go see it .  in fact , the entire theater applauded when the film ended , and laughs were so consistent , some jokes were probably missed due to the uproar .  the 1985 setting was milked for everything it was worth , and for those of us that are old enough to remember it , it was a bodacious trip back ( and done very well ) .  sandler shows a new side to himself while maintaining all his original charm ( or is it anti-charm ? ) .  while the wedding singer only deserves three stars due to artistic merit , grading it on pure enjoyability would give this one four stars .  it's hilarious , and it's touching .  it's got that great cheezy humor , but it also takes sandler to a higher level than anyone expected .  this is one to see again and again .  is it possible the wedding singer could be one of the sweetest romantic comedies of 1998 ?  my " magic 8-ball " predicts it is certain .   & 1 & The overall consensus is that the text expresses a highly positive sentiment towards the movie "The Wedding Singer." The reviewer loves the movie, praises the chemistry between the actors, gets emotional, enjoys the humor, and encourages others to watch it. The positive audience reaction at the end of the film is also mentioned. \\
1.000000 & as i walked out of crouching tiger , hidden dragon i thought to myself that i had had just seen a great film .  with the passage of a few hours i tempered my enthusiasm and started pondering the question of whether a masterpiece must implicitly be a " great " piece of work and viceversa .  attempting to make a distinction may be a matter of splitting hairs .  i avoided a commitment by appealing to etymology .  as the word implies , a masterpiece is any work which embodies the skills of a master .  as such it should suffice to say that it is a product of exceptional quality .  crouching tiger , hidden dragon fits comfortably in that category .  crouching tiger , hidden dragon immerses the viewer in an idealized world of oriental folklore , with the requisite blend of legend , fantasy , magic and mythology .  it is reminiscent of a wagnerian epic with characters which might as well be half-gods - greater than life , purer than life , stronger than life , physically invincible and able to accomplish superhuman feats , but with a human soul that makes them ultimately vulnerable .  all the classic elements of the oriental mystique are thrown into the mix , including the art of contemplation and the concept of martial dexterity as the physical equivalent of spiritual advancement .  the classic struggle between good and evil is the inevitable backdrop , with advanced masters of each and a golden pupil , which must choose between the two .  it is the gifted pupil who , under the influence of the evil master steals the holy grail in the form of a magic sword which is the focus of the conflict at the heart of the legend .  the elements of romance at two different levels of enlightenment ( a pair of masters and a pair of youngsters ) are poignantly represented .  the paradox of oriental restraint existing side by side with all consuming passion in the same breast is projected effectively .  the fight scenes are stunning balletic tours-de-force , not to be taken literally but clearly to be enjoyed as superb cinematic art , as are the prodigious leaps and flights to , from and between rooftops , the martial combat at the top of swaying bamboo branches and the combatants skipping like pebbles along the surface of a lake .  there need be no question of suspended disbelief when one is in the presence of poetry .  as in a wagnerian opera there is a substantial story line , which takes place at an ordinary human level , yet the entire project is to be accepted as a work of art rooted in fantasy .  do not assume , however , that the art is limp-wristed .  for those who are put off by the " art " label this film can be confidently recommended as engrossing entertainment at the levels of adventure , action and romance .  there are no weak performances in this movie .  michelle yeoh imbues her character with depth , humanity and wisdom .  chow yon fat projects dignity and purity of heart .  zhang ziyi is a budding superstar .  she is radiantly beautiful and totally persuasive in a multifaceted role .  cheng pei pei as the evil master and chang cheng as the bandit prince acquit themselves admirably .  kudos to screenwriters james schamus , wang hui ling and tsai kuo jing , choreographer yuen wo-ping , photography director peter pau and music director tan dun , each of which contribute quality components to this extraordinary film .  ang lee as the director , co-producer and mastermind of the project gets the lion share of the credit .  this is destined to be one of those films that everybody likes , including those who hate martial arts movies .  don't miss it .   & 1 & The film "Crouching Tiger, Hidden Dragon" is highly praised for its exceptional quality, captivating story, stunning fight scenes, and strong performances, making it a great and enjoyable film recommended to a wide range of viewers. \\
1.000000 & -- comedy , rated pg , runs about 1 : 40 -- starring : john goodman , kathy moriarty , and a bunch of teenagers -- directed by joe dante and written by charles hass --  summary : lawrence woolsley ( john goodman ) brings his new horror film mant !  to premiere in key west during the height of the cuban missile crisis .  he hopes to capitalize on the tense moment by providing an escape for the town .  we see most of the events through the stories of four teenagers and how life affects them .  quick and easy review : i really enjoyed matinee .  the mixture of comedy and tension blended nicely .  unlike many comedies this film tries , and succeeds , in getting past the stage of doing anything for a laugh .  the makers of the film also cared about telling an intelligent story .  the performances of all the principals are right on the mark , particularly john goodman as the schlock master .  so i would definitely recommend this film to anyone looking for a light hearted , yet interesting way to spend a couple of hours .  longer , more detailed review : [beware of spoilers]  the primary reason i enjoyed this film was , that while being a comedy , the film also had an intelligent story to tell .  too many comedies today subscribe to the the notion that a comedy need only make you laugh .  you watch the movie , laugh a lot , leave the theatre and take nothing with you .  matinee is not like that .  i left the picture thinking about what i would do faced with the cuban missile crisis .  i found myself wondering what would happen to the characters of the film .  but most importantly , i found myself caring about what would happen to the characters .  the comedy of the film centers around goodman , his character , and the film he brings to key west .  i believe that goodman is one of the finest comedic actors in the business today .  he is highly expressive both physically and vocally .  i felt he at least deserved an oscar nomination for his work in barton fink .  the other characters are often lost in a scene with him due to his commanding nature , however , while the star , goodman is actually not at the center of the film .  the movie is really the story of the four teenagers , discovering who they are and what they want , against a background where at any minute it could all end .  i thought the kids reaction were highly realistic .  they tried to block it out , they tried to escape from the concerns of their world .  unfortunately it kept creeping back in , particularly with the fear , and the chaos of the time .  while the comedy centered on goodman , and the drama on the teens , there was a great deal of overlap .  several aspects of the panic are shown in a humorous light .  one example is a scene where people are fighting each other for the the last cans and boxes of food in a grocery store .  if you think about it the threat of nuclear annihilation seems hardly to be the backdrop for a comedy , but it works here .  another reason i like this film is that i like b- science fiction movies .  one of my favorite films to go watch is plan 9 from outer space .   ( note i did not say it was one of my favorite movies , but one of my favorite to see . )  while mant !  never got made , many films like it were , and mant !  serves mostly as dante's homage to the b-films he loves .  so again i would like to recommend matinee to anybody looking for a good , humorous story .  this isn't a gag film like many other comedies but an intelligent , well-thought out , film about real people with real problems told in an often hilarious way .  enjoy !   & 1 & The overall sentiment towards the film "Matinee" is positive, as the reviewer enjoys the mixture of comedy and tension, appreciates the intelligent storytelling and performances of the actors, and highly recommends it for a light-hearted yet interesting experience. \\
1.000000 & wow ! what a movie .  it's everything a movie can be : funny , dramatic , interesting , weird , funny , weird and strikingly original .  yep that pretty much describes this movie .  it starts out like a regular movie and ends up being one of the weirdest , funniest most original movies i have ever seen .  it boggles the mind and some have to wonder why we cannot get movies like this more often .  besides being one of the best films of the year , being john malkovich may as well be one of the best movies ever .  period .  then again there are so many good movies , that one cannot pick an all time favorite .  john cusack plays a puppeteer craig schwartz a man out of a job , in search of a job .  his wife lotte schwartz who is played by a completely un-noticeable cameron diaz who looks like something off the streets is an animal lover and has about every kind of animal you could think of .  craig finds a job as a file at the 7 ? floor of a business building . . . you  have to pry open the elevator doors open before it reaches floor eight , the 7 ? floor is just a floor between 7 and 8 .  he is hired by his 105 year old boss ( orsen bean ) to be a filer .  in his office , he discovers a little door , to which was boarded up and hidden .  to his curiosity he opens it and starts to crawl toward it , he then gets sucked to the end and ends up in john malkovich's mind .  fifteen minutes later he is shot out onto the side of the new jersey turnpike .  he returns to tell his co-worker maxine ( catherine keener ) that he has found a portal that will lead him into john malkovich's mind , she doesn't believe him but after she sees it it changes her mind .  lotte also finds out about the portal and discovers that being someone else is good after all .  john malkovich of course has no idea what is going on , and by the end of this bizarre film there are so many twists and turns , that we don't know what really happened .  john cusack is outstanding and utterly believeable in a role only he himself could play .  he fits the role perfectly and to me was brilliantly cast .  cameron diaz is outstanding and utterly one of the world's most prettiest women , is made up here unnoticeable and very unattractive .  she however gives a very comic performance and this could easily be her best role to date .  catherine keener is very funny and sexy as maxine and of course the best thing of the movie is the magic himself john malkovich who is very brilliant and this movie plays big time homage ? to the master himself .  the whole group combined gives us a wonderfully funny movie that is also smart and clever .  spike jonze ( three kings ) makes his fabulous directing debut , and did a fantastic job of directing this new classic film .  he lets the viewer go on the trip as well as let the viewer know they are watching a movie .  by the time the ending rolled around my head was spinning from disbelief of how good this film was .  the screenplay written by charlie kaufman was hilarious and often thought-provoking .  the film also had a soft side to it and even though the ending is very surprising it is also a little sad and heart-warming .  the whole movie was fantastic and had me rolling in the isles .  from cameron diaz's appearence to john malkovich's explorations i laughed very hard , and it may as well be as funny as as good as it gets ( the funniest ! ) .  there isn't a slow point in the movie , or an overused idea .  there are no cliches except for the fact that this is the most original , inventie , witty , and smart movie i've seen in a long time .  i found myself amazed by everything : the direction , acting , writing and the whole idea of the movie .  by the end i had to wonder why hollywood doesn't want to make movies like this anymore .  or why they don't .  all filmmakers watch this movie and get some ideas of movies to come out .  this was a surprise hit as was american beauty .  'being john malkovich' is in the top 5 movies of the year , and in the top 10 best films ever made .  it has something that no other movies playing now has .  in fact it may as well be the best movie out right now .  i highly recommend 'being john malkovich' and have no doubts you will be disappointed .   & 1 & The summary explanation is that the movie is highly praised for being funny, dramatic, interesting, and original, with outstanding performances and fantastic directing, resulting in a very positive sentiment. \\
1.000000 & men in black is an explosive mix of science fiction , action , and comedy that hits the target in every possible way .  although another alien movie , men in black succeeds in every way that independence day didn't , and towers above many other movies of its type .  the brilliant acting , especially by tommy lee jones as agent kay , is also as good as it gets .  director barry sonnenfeld , who was behind the camera for the addams family movies and get shorty , has crafted a masterpiece .  the story behind men in black is just as interesting as you would want it to be .  the men in black , or mib , are a top-secret governmental agency that is not known to exist .  the mib are responsible for " saving the world from the scum of the universe " .  a though job , indeed .  the film opens with a truckload of illegal aliens ( the human kind ) being transported across the mexico border and into the united states .  presumably , these " aliens " are all migrant workers .  that is , until the mib show up and begin interrogating them .  agent kay selects a particular suspicious worker and takes him away from the other local authorities to discover that he is not an illegal human alien , but a real extra-terrestrial alien .  when the alien makes a run for it , agent kay is forced to eliminate the alien with one of the mib's very unique weapons , and after one of the local law enforcement officers witnesses this bizarre occurrence , agent kay is forced to use another very unique device on them .  the device , described as " out of state " , eliminates the memory of anyone it is used on .  >from here , we are introduced to james edwards , played very well by will smith .  edwards , a police officer , is chasing a fleeing criminal .  the criminal gives a very good chase , and at one point when edwards confronts him , the criminal pulls out a very different looking weapon that disintegrated when it hit the ground .  edwards continues to chase the very athletic criminal to the top of a building , where the criminal informs edwards that he must let him go , because someone is after him .  edwards doesn't take this seriously , but when the criminal shows very non-human characteristics and leaps off the building , he begins to wonder .  back at the police station , agent kay shows up to ask edwards a few questions .  he informs edwards that is was , in fact , a non-human that he was chasing , and that the gun he pulled out was definitely not man-made .  he has edwards identify the gun , and asks edwards to come to the mib headquarters the following day .  edwards arrives and finds that he is involved in a recruiting process , along with various other men who seem a bit more qualified than he .  after goofing up for half of the time , edwards puts on a show at the firing range , and agent kay notes the reason why he feels edwards should be the man to join the mib : he chased down the " criminal " on foot , which is something that no one is supposed to be able to do .  in the meantime , an upstate new york farm has a very close encounter .  edgar ( vincent d'onofrio ) , owner of the farm , investigates a strange crash landing and is attacked by the inhabitant of the flying object , which presumes to jump inside edgar and use his body as a human transport .  the " bug " , as he is called , is an intergalactic terrorist who has come to earth to attempt to kill two ambassadors .  and it up to the mib , with newly recruited agent jay ( formerly james edwards ) to exterminate the bug and save the planet from intergalactic war .  men in black delightfully combines fast-paced action with often hilarious comedy , which is usually from will smith , although tommy lee jones opens up his comedic personality in this film .  the special effects are also very well done and are not the entire source of the plot , as in another big alien film from the past summer .  screenwriter ed solomon , writer of super mario bros .  and the upcoming x-men film , has surely struck gold with this story .  all ages will enjoy men in black .  it is an extremely fun film that you will want to see again .  although it runs a very quick and speedy 96 minutes , the entire film from beginning to end is a non-stop adventure .  the ending of the film , which ties up a few loose ends for one of the main characters , is also very well done .  a sequel is already being planned , so there is more to look forward to !   & 1 & The explanations unanimously express a highly positive sentiment towards the movie "Men in Black," praising its combination of science fiction, action, and comedy, as well as the brilliant acting, masterful direction, interesting story, well-done special effects, and universal appeal to all ages. The review also expresses excitement for a sequel. \\
1.000000 & steven spielberg is now considered as one of the hollywood deities , because of the rare capability to deliver both huge commercial hits , like jurassic park , and " oscar " -awarded critical triumphs like schindler's list .  however , in the 1970s spielberg built his reputation by creating works of art that could slip in both categories .  one of them is close encounters of the third kind , extremely popular and influential science-fiction spectacle .  unfortunately , it had a bad luck to be released in the same year as star wars .  although both films have a lot in common ( ground-breaking special effects , brilliant score by john williams ) their future was different ; one became an unstoppable cult phenomenon , and another almost forgotten and stuck forever in its shadow .  when spielberg began work on that project , he was already established as a bright new hollywood star due to his previous commercial hit , jaws .  together with other young directors of his " new hollywood " generation , like kauffman , carpenter , hill and millius , he exploited the great creative freedom of 1970s , when even the mainstream producers dared to experiment .  ironically , it was spielberg himself whose later commercial success would established new unwritten rules of " blockubuster " philosophy .  but in the mid 1970s , many things were different ; spielberg was young and eager to use hollywood resources for his very personal and artistic movie .  although very personal , spielberg's screenplay was partly based on the book " ufo experience " by dr . j . allen hynek and in many ways inspired by the popular urban mythology of extraterrestrial visitors to earth that began to grow in the world after ww2 .  spielberg was not only inspired by the mythology , but his movie also gave the mythology itself a huge boost , unmatched until the contemporary era of x-files and the roswell anniversary .  that was partly because he made the movie very realistic using the authentic ufo-related incidents as the element of the plot .  the story begins with one of such incidents - team of international scientists come to the sonorra desert in mexico to find the u . s . navy planes of who went missing decades ago during the famous flight 19 .  such events coincide with the ufo incident witnessed by roy neary ( richard dreyfuss ) , power company worker from muncie , indiana , who later becomes obsessed with his experience .  because of his obsession he loses his job , family and sanity , but his loss is nothing compared to the experience of jillian guiler ( melinda dillon ) , single mother whose son becomes the victim of alien abduction .  in the meantime , the scientists decipher the strange signals from outer space and u . s . government , in co-operation with the french , led by lacombe ( francois truffaut ) begin with the preparation for ultra-secret project .  when the news of the poison gas leak in the middle of wyoming reach neary , he finally sees some sense in all his visions and begins the perilous journey toward the centre of endangered area .  there he is joined by jillian who shared the similar visions .  two of them must break through military pickets and reach their destination to find whatever is there .  spielberg here shows great mastery by using the very same techniques of jaws to make completely different effects .  the slow , gradual yet very disciplined series of dramatic incidents - " close encounters " - is set in order to bring the viewer to the great revelation in the finale .  but , instead of the fear and horror we had to endure during the jaws , we are now overwhelmed by the sense of boyish wonder .  throughout the movie the viewer knows that something big , magnificent and wonderful is about to happen , and great magician spielberg delivers his promise in the end .  the last sequence , with its , even in this age , impressive special effects by the great virtuoso douglas trumbull , would leave many mouths open .  one of the great virtues of this film is its optimism .  aliens , who almost always get portrayed as the monsters in science-fiction cinema , are here benevolent and harmless creatures and the first contact between them and humanity is a beginning of something wonderful .  it is very ironic , when we consider that the two classic sf movies that visually inspired spielberg actually told quite different story - howard hawks' thing and byron haskin's war of the worlds presented extraterrestrials as the threat to the mankind .  spielberg's humane approach and faith in the future also lies in great contrast to the pessimistic mood of its era ; the only hint of the contemporary gloom is post-watergate portrayal of government as conspiratorial towards the public .  but , even such government is much more harmless compared to the murderous and chain-smoking men in black that became the stereotype thanks to x-files and its more cynical and disturbing visions .  there lies the main , and probably the only flaw of this great picture - lack of conflict , and consequently , lack of drama .  the movie has few excitements or even action scenes ( especially the last that may be an interesting homage to hitchcock's north by northwest ) but generally , almost everyone - neary , jillian , government , aliens - are the good guys .  despite such shortcomings , the actors were good and manage to bring multidimensionality to their simple roles .  richard dreyfuss is very convincing as a ordinary , yet nice guy , who sinks into insanity only to rediscover himself in a grand finale .  melinda dillon was , on the other hand , nominated for " oscar " as a struggling mother , yet she was overshadowed by teri garr as neary's long-suffering wife ronnie .  apart from visual wonders of this film , spielberg's semi-official composer john williams again excels by his beautiful music , this time using the simple melody both as the element of a plot , and as the basis for his score .  the aliens , who are the main subject of this film , were visually very convincing .  too convincing , one of my acquaintances in the ufo-researching circles said .  according to him , the depiction of extraterrestrials as grey-skinned little people with big eyes was so accurate , that it managed to freak out powerful government figures interested in suppressing the truth about ufos .  so , they later approached spielberg and ordered him to make another movie with alien , this time designed to be anything but the real life .  the result was e . t . , for many years the biggest commercial hit of all times , yet less inspirational for ufo enthusiasts .  anyway , whether the viewer believes in existence of extraterrestrials or ufos , close encounters of the third kind remains the great movie , and one of the rare uplifting experiences in modern cinema .   & 1 & The overall sentiment of the explanations is positive, as they discuss Steven Spielberg's talent, influence, and successful films, particularly "Close Encounters of the Third Kind." \\
1.000000 & martin scorsese's triumphant adaptation of edith wharton's the age of innocence is a stunning film for the quintessential new york filmmaker , the man who brought the streets of taxi driver and mean streets to life .  it seems like an odd choice for scorsese to do a period piece in the early 1900's , but the fact that he pulls it off so brilliantly is a wonder , and a testament to the greatness of scorsese as a filmmaker .  this is a gorgeous visual experience that it surely one of scorsese's finest .  newland archer ( day-lewis ) is a prestigious lawyer who is engaged to may welland ( ryder ) , a somewhat empty and shallow new yorker , who belongs to a prestigious family and is quite beautiful .  the marriage is one which will unite two very prestigious families , in a society where nothing is more important than the opinions of others .  on the day that archer is to announce his engagement to may , countess ellen olenska ( pfeiffer ) , cousin of may , walks into archer's life .  archer is immediately captivated , and finds himself in love with ellen .  archer is also bound by the limits of new york society , which is an intrusive as any other in the world .  archer finds himself having a secret love affair in his mind with countess olenska , attempting to keep her in his mind while trying not to lose his social status .  the film's subject matter may seem alien to scorsese , but the theme is definitely not .  it is a theme of forbidden romance , guilty pleasures , and the consequences causes because of those actions .  there is a painstakingly flawed hero , and his choice between the life he wants , and the life he is destined for .  in truth , it is a film about a society the audience doesn't know about , but wants to find out more , much like the society of goodfellas or even kundun .  the performances are absolutely breathtaking .  day-lewis portrays more mental anguish in his face than one man should be forced to take .  pfeiffer is marvelous as countess olenska , a mix of passion and beauty that the audience would die for as well .  ryder is probably the gem of the group , for it is her quiet presence that overwhelms the plot , and slowly pushes day-lewis closer and closer to his eventual ending .  the supporting cast is also wonderful , with several characters so singular that they are indelible in one's memory .  scorsese definitely has a passion for filmmaking .  his lavish and sumptuous set design and marvelous recreation of new york is a wondrous sight .  he literally transports the viewer to another world with incredible imagery .  his script is also excellent , slow in buildup , with a rapid conclusion and a fantastic ending that has to be seen to be believed .  it is difficult to make a period piece gripping : scorsese , however , does it beautifully .  the famous cameras of the legendary director are also everywhere .  he is patient , but he films everything and anything remotely important .  the cameras sweep , pan , track , and do more than they've ever done , but they are so subtle , one doesn't realize he's watching all the scorsese hallmarks until a 2nd viewing .  the central tracking shot is probably longer and more complex than the famous goodfellas shot , but the viewer doesn't notice it , because we want to see more of this gorgeous world .  there are a few deft touches of filmmaking that are simply outstanding , and joanne woodward' narration is exquisite .  not a fast film like goodfellas , this shares more in common with kundun than anything else .  and like kundun , this is a slow-starting film that truly shines , when given the chance to fully breathe and bloom in the end .  a beautiful film by a director continuing to challenge himself year after year .   & 1 & The user's prompt expresses overwhelmingly positive sentiment towards Martin Scorsese's film "The Age of Innocence." They praise Scorsese's ability to successfully adapt a period piece, complimenting the performances, visuals, set design, script, and overall filmmaking techniques. The reviewer's enthusiastic tone and use of positive adjectives contribute to the overall highly positive sentiment of the text. \\
1.000000 & capsule : this is a 1950s or 1960s style heist film , set in the present .  robert deniro stars as a risk-adverse safecracker who wants to retire form crime but takes one last job at the request of a personal friend ( played by marlon brando ) .  edward norton plays a hotshot young sharpster who is also in on the crime .  the plot is mostly straightforward suspense with little nonsense .   , +2 ( -4 to +4 )  i am sure i must have seen almost the identical plot before .  this is a heist film made for an adult audience who probably wanted a crime film like they had seen in theaters when they were teens .  there are no superhuman acrobats taking nosedives off of buildings like in entrapment .  there is no rock score .  there are no ballet- like martial arts .  this is just a basic heist film with a decent and distinctly credible and un-flashy script .  nick ( played by robert deniro ) is a safecracker who has managed to be successful by never taking risks .  if a job is not a safe bet ( pun intended ) , he backs out .  sometimes even the safe bets turn out not to be so safe .  when one job very nearly goes wrong nick is unnerved enough to decide that it is nature telling him that it is time to get out of the game .  he returns to his home in montreal where he owns a jazz club , and decides to manage it full time .  he proposes to his girl friend diane ( angela bassett ) .  she has one condition .  he must stay retired from crime .  but before the deal can be cemented , max , a montreal kingpin and personal friend , has one last supposedly easy job for nick .  nick wants no part particularly because the heist will be right in his hometown of montreal .  more and more details seem to complicate the job .  nick's partner in the crime is to be a smart , but uncontrollable young crook , jack ( edward norton ) .  jack treats a locked front door like a welcome mat , even at his associates' homes .  the young crook is a know-it-all who seems good at everything he does but at avoiding rubbing people the wrong way .  together they plan to steal a priceless historic artifact from the montreal customs house .  the script by kario salem , lem dobbs , and scott marshall smith works like an episode of the old " mission impossible " television series .  we see pieces of the heist being put together , last minute changes , and things that go wrong , much like a good episode of " mission impossible . "  this team might not be bad choices to write scripts for the tom cruise " mission impossible " films .  the complications are , however no more and no fewer than are needed to make the story believable .  the telling is cold and noirish , which is just what it is supposed to be .  director frank oz , the voices of yoda and miss piggy proves surprisingly good at directing a serious crime film .  the score has a more than adequate cast with little flashy or scene-stealing acting .  edward norton probably has the flashiest role and even that is low-key by today's standards .  he plays what is nearly a double role .  jack pretends to be a brain damage victim to be hired for a job in the customs house .  one nice ( ? )  character i have not mentioned is stephen ( jamie harrold ) .  stephen is a master hacker who lives in his mother's basement in a house with a lot of screaming in both directions .  he seems like the last person the risk adverse nick would want to depend upon .  the film itself remains low-key up until the time of the climactic heist .  then the pace really picks up .  before that the plot even stops twice for jazz interludes .  though oz never lets the music steal time from the story the way woody allen does in sweet and lowdown .  on the subject of music , the score of the score is by howard shore .  it adds tension to the suspense scenes , but never seems to have much of a melody .  angela bassett is the one misused celebrity in a totally minor role that should have been played by a less famous actress who needed a break .  she has nothing to do in the film but demand that nick give up crime and to look like an attractive reward if he does .  speaking of being attractive the score seems to be attracting an older audience who learned to appreciate much the same sort of film in the 1950s and 1960s .  it does the job .  i rate it a 7 on the 0 to 10 scale and a +2 on the -4 to +4 scale .   & 1 & The film "Capsule" is generally viewed positively, with praises for its straightforward suspense, credible script, directing, and acting. However, some reservations are mentioned regarding the lack of originality in the plot and a forgettable score. \\
1.000000 & being that it is a foreign language film with no known names with a select number theaters showing it , " shall we dance " won't be seen by that many people .  and that's a shame - this is a funny , enchanting , and goofy movie full of laughs , surprises , and wonderful dance sequences .  the surprising thing about " shall we dance " is the universal appeal of the story .  a us version is in the works , and it's no wonder - it's not really all that culture specific .  although there is a narrated set-up that adds some extra resonance to the proceedings ( about the view the japanese culture has about ballroom dancing ) , the movie is so rich in character and appeal that this added layer isn't at all necessary to understand or enjoy the film .  the only important unexplained japanese-specific reference that may leave some a bit puzzled is that 1000 yen is roughly $10 ( when you see the film - and you should see this film - you'll know why ) .  a middle aged company man realizes that achieving all the goals he set for himself in life ( a house , a child , and good marriage ) still doesn't translate into a fulfilled life .  after glimpsing a melancholy beauty looking out from a dance studio window while on the train home from work , sugiyana ( koji yakusho ) decides after some trepidation to take up ballroom dance lessons in order to meet the woman who has stirred something in himself .  after finding the weekly fees for private lessons from the elegant and beautiful mai ( tamiyo kusakari ) too rich for his blood , sugiyama opts for groups lessons simply to be near her .  we then meet the players in this gem of a movie , who all have their own reasons for joining the class .  later on , we meet aoki at the dance studio ( naoto takenaka ) , a co-worker and all out weirdo and one of the laugh riot highlights of the film .  the story , laughs , and touching scenes evolve as the movie goes along .  it's a pleasure to watch such a wonderful film that is propelled almost solely by the characters and performances .  it's difficult to explain the charms of the film without revealing too much - the movie abounds with little revelations that subtly shape the characters , and in the end , each one is that much fuller and more understood by the time you leave the theater .  it's one of those movies that only the french seem to make anymore - no big plot , no special effects , no gunplay , no tragic consequences , no forced examinations of the nature of love , and no insights into the nature of evil .  just a warm , funny , endearing film that will charm the pants off of you .  when was the last time you left a theater feeling all warm and fuzzy inside ?   " shall we dance " will do that to you without any treacly aftertaste .   & 1 & The text consistently praises the film "Shall We Dance" as highly positive, describing it as funny, enchanting, goofy, and having wonderful dance sequences. It emphasizes the universal appeal of the story, the rich characters, and performances, and how the film leaves the audience feeling warm and fuzzy inside. Overall, the reviewer strongly recommends watching the film with no negative aspects mentioned. \\
\bottomrule
\end{tabular}
